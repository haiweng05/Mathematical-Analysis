\documentclass{article}
\usepackage{graphicx} % Required for inserting images
\usepackage{ctex}

\usepackage{amsfonts}
\usepackage{amssymb}
\usepackage{amsmath}
\usepackage{amsthm}
% 导入 xparse 宏包以支持 LaTeX3 语法
\usepackage{xparse}
\usepackage{pgfplots}
% 用于插入带有坐标轴、标签和曲线的图像

% \begin{tikzpicture}
%     \begin{axis}[
%       xlabel=$x$,
%       ylabel=$f(x)$,
%       axis lines=middle,
%       xmin=-5, xmax=5,
%       ymin=-2, ymax=8,
%       width=\textwidth,
%       height=8cm
%     ]
%     \addplot[blue,domain=-3:3] {x^2};
%     \end{axis}
%   \end{tikzpicture}
\usepackage{tikz}
% 用于绘制一般的图像

% \begin{tikzpicture}[scale=0.8]
%     \draw[->] (-4,0) -- (4,0) node[right] {$x$};
%     \draw[->] (0,-1) -- (0,9) node[above] {$f(x)$};
%     \draw[domain=-3:3,smooth,variable=\x,blue] plot ({\x},{\x^2});
%   \end{tikzpicture}

\newtheorem{theorem}{Theorem}[subsection]
\newtheorem{lemma}{Lemma}[subsection]
\newtheorem{corollary}{corollary}[subsection]
\newtheorem{example}{Example}[subsection]
\newtheorem{definition}{Definition}[subsection]
% 为了证明中可以使用中文,后续定义证明时使用cproof而不是proof
\newenvironment{cproof}{%
{
    \textbf{Proof\\}
    }
}{
%   \hfill $\square$ 添加结束符号
%   \par\bigskip 可选的垂直间距
}
\newenvironment{exercise}[1]{%
{\textbf{Exercise #1} \\ 
    }
}{
  \hfill $\square$ 
  \par\bigskip 
}

\newenvironment{solution}{%
{
    \textbf{Solution\\}
    }
}{
  \hfill $\square$ 
  \par\bigskip 
}


% \newenvironment{identification}{%
% \heiti{定义}\kaishu
% }{%
% %   \hfill $\square$ 添加结束符号
% %   \par\bigskip 可选的垂直间距
% }

\newcommand{\RR}{\mathbb{R}}
\newcommand{\NN}{\mathbb{N}}
\newcommand{\CC}{\mathbb{C}}
\newcommand{\QQ}{\mathbb{Q}}
\newcommand{\ZZ}{\mathbb{Z}}
\newcommand{\FF}{\mathbb{F}}
\newcommand{\PP}{\mathbb{P}}
% 简化各种常见数的集合

\newcommand{\parameter}[1]{\left(#1\right)}

\newcommand{\bracket}[1]{\left[#1\right]}

\newcommand{\abs}[1]{\left|#1\right|}
% 各种自动变化大小的括号的简化

\newcommand{\ve}{\boldsymbol}
% 为了适应David C Lay线性代数中,简化斜体+粗体向量的书写

\newcommand{\base}{\mathcal}

\newcommand{\tb}{\textbf}

\newcommand{\col}{\text{Col}}

\newcommand{\row}{\text{Row}}
\newcommand{\nul}{\text{Nul}}
\newcommand{\spans}{\text{Span}}
\newcommand{\proj}{\text{proj}}
\newcommand{\adj}{\text{adj.}}
\newcommand{\rank}{\text{rank}}
\newcommand{\range}{\text{range}}
\newcommand{\n}{\text{null}}
\newcommand{\tr}{\text{tr}}
\newcommand{\sign}{\text{sign}}
\newcommand{\perm}{\text{perm}}
% 简化粗体字体的书写

\newcommand{\f}[2]{\frac}
\newcommand{\df}[2]{\dfrac}

\newcommand{\ip}[1]{\left<#1\right>}

\newcommand{\pa}{\paragraph}
\newcommand{\spa}{\subparagraph}
\newcommand{\se}{\section}
\newcommand{\sse}{\subsection}
\newcommand{\ssse}{\subsubsection}

% \NewDocumentCommand{\vs}{m m m}{
%     \ve{#1}_{#2},\cdots,\ve{#1}_{#3}
% }
% % 快速书写一个向量组,第一个参数为向量名称,后两个为首末角标

% % $\vs{b}{1}{n} $这是多参数命令的使用示例

% \NewDocumentCommand{\cvs}{m m m m m}{
%     #1_{#4}\ve{#2}_{#4} #3 \cdots #3 #1_{#5}\ve{#2}_{#5}
% }
% % 快速书写一个向量线性组合,第一个参数为系数,第二个参数为向量名称,第三个参数为运算符,后两个参数为角标

\NewDocumentCommand{\size}{m m}{
    #1\times#2
}


\title{习题四}
\author{徐海翁}
\date{2024.3.13}

\begin{document}

\begin{CJK}{UTF8}{gkai}

\maketitle
\tableofcontents

\begin{exercise}{8.5}
    (1)
    \[\lim_{x\to +\infty}\dfrac{x^{\frac{5}{4}}}{\sqrt[4]{3x^5 + 2x^2 - 1}} = \lim_{x\to +\infty}\dfrac{1}{\sqrt[4]{3}}\]
    根据无穷积分的比较判别法,可知收敛

    (2)
    \[\int_{0}^{+\infty}\dfrac{\sin^2 x}{x^2}\, dx  = \int_{0}^{\frac{\pi}{2}}\dfrac{\sin^2 x}{x^2} \, dx + \int_{\frac{\pi}{2}}^{+\infty}\dfrac{\sin^2 x}{x^2}\, dx \]
    其中第一项是瑕积分,满足
    \[\lim_{x \to 0}\dfrac{\sin^2 x}{x^2} \cdot x^p = 0,\forall p \in (0,1)\]
    故根据比较判别法知其收敛;

    其中第二项无穷限积分满足
    \[\int_{\frac{\pi}{2}}^{+\infty}\dfrac{\sin^2 x}{x^2}\, dx \leq \int_{\frac{\pi}{2}}^{+\infty}\dfrac{1}{x^2}\, dx \]
    也是收敛的,故这个积分收敛

    (3)由变量代换$e^t = x$可知
    \[\int_{0}^{1}\frac{1}{\ln x}\, dx =\int_{-\infty}^{0}\frac{e^t}{t}\, dt = \int_{-\infty}^{-1}\frac{e^t}{t}\, dt + \int_{-1}^{0}\frac{e^t}{t}\, dt\]
    由于对$\forall p > 1$我们都有
    \[\lim_{-\infty}\frac{e^t}{t} t^p = 0\]
    从而我们知道上面的第一项收敛,然而对于第二项我们有
    \[\int_{-1}^{0}\frac{e^t}{t}\, dt > \int_{-1}^{0}\frac{1}{e} \frac{1}{t}\, dt\]
    是发散的,故这个积分发散
\end{exercise}

\begin{exercise}{8.6}
    (1)当$p > 1$时我们有
    \[\int_{1}^{+\infty}\frac{\cos x}{x^p}\, dx \leq \int_{1}^{+\infty}\frac{1}{x^p}\]
    收敛,故$p > 1$时绝对收敛;

    当$0 < p \leq 1$时,我们利用分部积分
    \[\int_{1}^{+\infty}\frac{\cos x}{x^p}\, dx  = (\frac{\sin x}{x^p})|_1^{+\infty} + p\int_{1}^{+\infty}\frac{\sin x}{x^{p + 1}}\]
    其中第一项收敛,第二项可以根据类似于$p > 1$情况的讨论证明其同样收敛,故此时原积分收敛.

    然而
    \[\int_{1}^{+\infty}\abs{\frac{\cos x}{x^p}}\, dx \geq \int_{1}^{+\infty}\frac{\cos^2 x}{x^p}\, dx = \int_{1}^{+\infty}\frac{1}{2x^p}\, dx + \int_{1}^{+\infty}\frac{\cos 2x}{2x^p}\, dx\]
    其中第一项发散,第二项收敛,故原积分并非绝对收敛,其条件收敛;\\

    综上,$0< p \leq 1$时,原积分条件收敛,$p > 1$时,原积分绝对收敛\\

    (2)我们记$f(x) = \dfrac{\ln \ln x}{\ln x}, g(x) = \sin x$,其中对于$f(x)$求导可得
    \[f'(x) = \dfrac{1 - \ln \ln t}{t(\ln t)^2}\]
    从而有$f(x)$在$[e^e,\infty)$单调递减,且利用洛必达法则可知
    \[\lim_{x\to +\infty} f(x) = \lim_{x\to +\infty}\frac{1}{\ln x} = 0\]
    又因为$\abs{\int_{e^e}^{+\infty}g(x)\, dx} \leq 2$有界,从而利用迪利克雷判别法可知
    \[\int_{e^e}^{+\infty}\frac{\ln\ln x}{\ln x}\sin x \, dx\]收敛.
    又由于常规黎曼积分
    \[\int_{3}^{e^e}\frac{\ln\ln x}{\ln x}\sin x \, dx\]
    显然收敛,故我们的积分收敛;

    而
    \[\int_{e^e}^{\infty}\abs{\frac{\ln\ln x}{\ln x}\sin x }\, dx \geq \int_{e^e}^{\infty}\frac{\ln\ln x}{\ln x}\sin^2 x \, dx = \int_{e^e}^{\infty}\frac{\ln\ln x}{2\ln x} \, dx - \int_{e^e}^{\infty}\frac{\ln\ln x}{2\ln x}\cos 2x \, dx\]
    其中第一项有
    \[\frac{\ln \ln x}{\ln x} > \frac{1}{\ln x}  > \frac{1}{x}\]
    对$x > e^e$恒成立,故第一项积分不收敛,第二项积分可以类似前面的讨论利用迪利克雷判别法证明其收敛.从而
    \[\int_{e^e}^{\infty}\abs{\frac{\ln\ln x}{\ln x}\sin x }\, dx\]
    不收敛,那么加上一段区间后的
    \[\int_{3}^{\infty}\abs{\frac{\ln\ln x}{\ln x}\sin x }\, dx\]
    同样不收敛;

    综上,原积分条件收敛;
\end{exercise}

\begin{exercise}{8.7}
    由于$\lim_{x \to 0^+} f(x) = +\infty$且有$f(x)$在区间单调,我们可知$f(x)$单调递减,并且利用这个极限可知$\forall M > 0$,$\exists \delta >0$使得$x < \delta$时有$f(x) > M > 0$

    由于$\int_{0}^{1}f(x)\, dx$收敛,由广义积分的柯西收敛原理,$\exists H > 0$使得$\forall 0 < x ,x' < \min{H,\delta}$有$0 < \int_{x}^{x'}f(x)\, dx < \varepsilon$,不妨取$x' = h < \min{H,\delta}$,$x = \frac{h}{2}$,则
    \[\varepsilon > \int_{\frac{h}{2}}^{h}f(x)\, dx > \frac{h}{2}f(h) > 0\]
    取$h\to 0^+$,我们得到
    \[\lim_{x\to 0^+}xf(x) = 2 \lim_{h \to 0^+} \frac{h}{2}f(h) = 0\]
\end{exercise}

\begin{exercise}{8.8}
    (1)
    \[\int_{0}^{1}x^{p - 1}(1 - x)^{q - 1}\, dx = \int_{0}^{0.5}\frac{x^p}{x}\frac{(1 - x)^q}{1 - q}\, dx + \int_{0.5}^{1}\frac{x^p}{x}\frac{(1 - x)^q}{1 - x}\, dx\]
    由于
    \[\lim_{x \to 0} \frac{x^p\cdot x}{x}\frac{(1 - x)^q}{1 - x} = 0\]
    上式的第一项收敛,同理
    \[\lim_{x \to 1} \frac{x^p}{x}\frac{(1 - x)^q\cdot (1 - x)}{1 - x} = 0\]
    故上式第二项收敛,故整个式子收敛\\

    (2)
    利用变量代换$t = \ln x$,
    \[\int_{0}^{1}|\ln x|^p \, dx = \int_{-\infty}^{0}|t|^p e^t\, dt\]
    由于$\forall q > 1$
    \[\lim_{t\to - \infty}|t|^p e^t t^q = 0\]
    利用无穷限积分的比较判别法,可知原式子收敛\\

    (3)利用变量代换$1 - x = e^t$,则
    \[\int_{0}^{1}\frac{\ln (1 - x)}{1 - x^2} \, dx = \int_{0}^{-\infty}\frac{t}{2 - e^t}\, dt\]
    由于$\lim_{t \to -\infty} \frac{t^2}{2 - e^t} = + \infty$,根据无穷限积分的比较判别法,原式子不收敛
\end{exercise}

\begin{exercise}{8.9}
    \[
        \begin{aligned}
            \int_{0}^{\frac{\pi}{2}} \frac{dx}{\sin^p x \cos^q x} &= \int_{0}^{\frac{\pi}{4}} \frac{dx}{\sin^p x \cos^q x} + \int_{\frac{\pi}{4}}^{\frac{\pi}{2}} \frac{dx}{\cos^p (\frac{\pi}{2}- x) \sin^q (\frac{\pi}{2}- x)}\\
            &= \int_{0}^{\frac{\pi}{4}} \frac{dx}{\sin^p x \cos^q x} + \int_{0}^{\frac{\pi}{4}} \frac{dx}{\cos^p x \sin^q x}\\
        \end{aligned}
    \]
    $p < 1,q < 1$时,我们有
    \[\lim_{x\to 0} \frac{x^p}{\sin^p x \cos^q x} = 0\]
    \[\lim_{x\to 0} \frac{x^q}{\sin^q x \cos^p x} = 0\]
    故上式中两项均收敛,原积分收敛;由于此时被积函数恒大于0,我们有原积分绝对收敛.

    $p \geq 1$或$q \geq 1$时,两项中至少一项发散到正无穷,且未发散的项也一定为正,故原积分发散.

    综上$p < 1,q < 1$时,原积分绝对收敛,$p \geq 1$或$q \geq 1$时,原积分发散
\end{exercise}

\begin{exercise}{9.1}
    (1)\[Int A = (0,1)\cap (3,4), \partial A = \{0,1,2,3,4,5\}, \bar{A} = [0,1]\cap[3,4]\cap\{2,5\}\]

    (2)\[Int B = \varnothing, \partial B = B = \{\sqrt{2}m + n|m,n \in \ZZ\},\bar{B} = B = \{\sqrt{2}m + n|m,n \in \ZZ\}\]

    (3)\[Int C = \varnothing, \partial C = \{2\}\cap\left\{2 + \frac{1}{n}|n \in \NN\right\},\bar{C} = \{2\}\cap\left\{2 + \frac{1}{n}|n \in \NN\right\}\]
\end{exercise}

\begin{exercise}{9.2}
    (1)不可以,我们可以逐一验证以上性质,首先正定性是显然成立的,其次交换性由于平方和绝对值的存在也是满足的,然而并不满足三角恒等式,例如$x = (1,0),y = (2,0),z = (0,0)$

    那么有
    \[d(x,y) = 1,d(x,z) = 1,d(y,z) = 4\]
    这与三角不等式的要求
    \[4 = d(y,z) \leq d(x,z) + d(x,y) = 2\]
    矛盾

    (2)可以.首先正定性可以通过平方根的存在直接判断,其次交换性由于根号内平方的存在也成立,最后是三角不等式,我们如下证明

    我们可以通过柯西不等式导出
    \[\begin{aligned}
        \sqrt{(x_1 - z_1)^2 + 4(x_2 - z_2)^2} &\sqrt{(y_1 - z_1)^2 + 4(y_2 - z_2)^2}\\
        &\geq (z_1 - x_1)(y_1 - z_1) + 4(z_2 - x_2)(y_2 - z_2)\\
    \end{aligned} 
    \]
    也就有
    \[d(x,z) d(y,z) \geq -\frac{1}{2}(d^2(x,z) + d^2(y,z) - d^2(x,y))\]
    移项也就有
    \[d^2(x,z) + d^2(y,z) - d^2(x,y) + 2 d(x,z) d(y,z) \geq 0\]
    即
    \[d(x,y) \leq d(x,z) + d(y,z)\]
    以下是这个距离定义下的单位圆,如果写成解析式即是$x_1^2 + 4x_2^2 = 1$
    \begin{center}
        \begin{tikzpicture}
            \begin{axis}[
              axis lines=middle,
              xlabel=$x_1$,
              ylabel=$x_2$,
                xmin=-2, xmax=2,
                ymin=-1, ymax=1,
                width=\textwidth,
                height=8cm
            ]
              \addplot [domain=-2:2,samples = 1000] {sqrt((1 - x^2)/4)};
              \addplot [domain=-2:2,samples = 1000] {-sqrt((1 - x^2)/4)};
            \end{axis}
          \end{tikzpicture}
    \end{center}

    (3)不可以,例如对于$x = (1,2),y = (1,3)$,二者满足$d(x,y) = 0$但$x \neq y$,这与距离的性质矛盾

    (4)不可以,例如对于$x = (1,2),y = (1,1)$,二者满足$d(x,y) = 0$但$x \neq y$,这与距离的性质矛盾
\end{exercise}

\end{CJK}
\end{document}

\begin{itemize} 
\end{itemize}