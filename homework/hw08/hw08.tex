\documentclass[11pt,hyperref,a4paper,UTF8]{ctexart}
\usepackage[left=2.50cm, right=2.50cm, top=2.50cm, bottom=2.50cm]{geometry}
\usepackage[unicode=true,colorlinks,urlcolor=blue,linkcolor=blue,bookmarksnumbered=true]{hyperref}
\usepackage{graphicx} % Required for inserting images
\usepackage{ctex}

\usepackage{amsfonts}
\usepackage{amssymb}
\usepackage{amsmath}
\usepackage{amsthm}
% 导入 xparse 宏包以支持 LaTeX3 语法
\usepackage{xparse}
\usepackage{pgfplots}
% 用于插入带有坐标轴、标签和曲线的图像

% \begin{tikzpicture}
%     \begin{axis}[
%       xlabel=$x$,
%       ylabel=$f(x)$,
%       axis lines=middle,
%       xmin=-5, xmax=5,
%       ymin=-2, ymax=8,
%       width=\textwidth,
%       height=8cm
%     ]
%     \addplot[blue,domain=-3:3] {x^2};
%     \end{axis}
%   \end{tikzpicture}
\usepackage{tikz}
% 用于绘制一般的图像

% \begin{tikzpicture}[scale=0.8]
%     \draw[->] (-4,0) -- (4,0) node[right] {$x$};
%     \draw[->] (0,-1) -- (0,9) node[above] {$f(x)$};
%     \draw[domain=-3:3,smooth,variable=\x,blue] plot ({\x},{\x^2});
%   \end{tikzpicture}

\newtheorem{theorem}{Theorem}[subsection]
\newtheorem{lemma}{Lemma}[subsection]
\newtheorem{corollary}{corollary}[subsection]
\newtheorem{example}{Example}[subsection]
\newtheorem{definition}{Definition}[subsection]
% 为了证明中可以使用中文,后续定义证明时使用cproof而不是proof
\newenvironment{cproof}{%
{
    \textbf{Proof\\}
    }
}{
%   \hfill $\square$ 添加结束符号
%   \par\bigskip 可选的垂直间距
}
\newenvironment{exercise}[1]{%
{\textbf{Exercise #1} \\ 
    }
}{
  \hfill $\square$ 
  \par\bigskip 
}

\newenvironment{solution}{%
{
    \textbf{Solution\\}
    }
}{
  \hfill $\square$ 
  \par\bigskip 
}


% \newenvironment{identification}{%
% \heiti{定义}\kaishu
% }{%
% %   \hfill $\square$ 添加结束符号
% %   \par\bigskip 可选的垂直间距
% }

\newcommand{\RR}{\mathbb{R}}
\newcommand{\NN}{\mathbb{N}}
\newcommand{\CC}{\mathbb{C}}
\newcommand{\QQ}{\mathbb{Q}}
\newcommand{\ZZ}{\mathbb{Z}}
\newcommand{\FF}{\mathbb{F}}
\newcommand{\PP}{\mathbb{P}}
% 简化各种常见数的集合

\newcommand{\parameter}[1]{\left(#1\right)}

\newcommand{\bracket}[1]{\left[#1\right]}

\newcommand{\abs}[1]{\left|#1\right|}
% 各种自动变化大小的括号的简化

\newcommand{\ve}{\boldsymbol}
% 为了适应David C Lay线性代数中,简化斜体+粗体向量的书写

\newcommand{\base}{\mathcal}

\newcommand{\tb}{\textbf}

\newcommand{\col}{\text{Col}}

\newcommand{\row}{\text{Row}}
\newcommand{\nul}{\text{Nul}}
\newcommand{\spans}{\text{Span}}
\newcommand{\proj}{\text{proj}}
\newcommand{\adj}{\text{adj.}}
\newcommand{\rank}{\text{rank}}
\newcommand{\range}{\text{range}}
\newcommand{\n}{\text{null}}
\newcommand{\tr}{\text{tr}}
\newcommand{\sign}{\text{sign}}
\newcommand{\perm}{\text{perm}}
% 简化粗体字体的书写

\newcommand{\f}[2]{\frac}
\newcommand{\df}[2]{\dfrac}

\newcommand{\ip}[1]{\left<#1\right>}

\newcommand{\pa}{\paragraph}
\newcommand{\spa}{\subparagraph}
\newcommand{\se}{\section}
\newcommand{\sse}{\subsection}
\newcommand{\ssse}{\subsubsection}

% \NewDocumentCommand{\vs}{m m m}{
%     \ve{#1}_{#2},\cdots,\ve{#1}_{#3}
% }
% % 快速书写一个向量组,第一个参数为向量名称,后两个为首末角标

% % $\vs{b}{1}{n} $这是多参数命令的使用示例

% \NewDocumentCommand{\cvs}{m m m m m}{
%     #1_{#4}\ve{#2}_{#4} #3 \cdots #3 #1_{#5}\ve{#2}_{#5}
% }
% % 快速书写一个向量线性组合,第一个参数为系数,第二个参数为向量名称,第三个参数为运算符,后两个参数为角标

\NewDocumentCommand{\size}{m m}{
    #1\times#2
}


\title{习题七}
\author{徐海翁}
\date{2024.3.27}

\begin{document}

\begin{CJK}{UTF8}{gkai}

\maketitle
% \tableofcontents

\begin{exercise}{21}
    题目有误,将$\xi,\eta$表达式中的$u$改成$y$.

    于是我们由链式法则可得
    \[\frac{\partial u}{\partial x} = \frac{\partial u}{\partial \xi} \frac{\partial \xi}{\partial x} + \frac{\partial u}{\partial \eta} \frac{\partial \eta}{\partial x} = u_\xi + u_\eta\]

    同理
    \[\frac{\partial u}{\partial y} = \lambda \frac{\partial u}{\partial \xi} + \mu \frac{\partial u}{\partial \eta}\]

    进而可以求得各二阶偏导
    \[u_{xx} = u_{\xi\xi} + u_{\eta\eta} + u_{\eta\xi} + u_{\xi\eta}\]
    \[u_{xy} = \lambda u_{\xi\xi} +\mu u_{\eta\eta} +\lambda u_{\eta\xi} +\mu u_{\xi\eta}\]
    \[u_{yy} = \lambda^2 u_{\xi\xi} +\mu^2 u_{\eta\eta} +\lambda \mu u_{\eta\xi} +\lambda \mu u_{\xi\eta}\]

    由于
    \[u_{xx} + au_{xy} + bu_{yy} = 0\]

    不妨取$\lambda = 0$,注意到此时$\mu \neq 0$,那么
    \[u_{\xi\xi} + (1 + a\mu + b\mu^2) u_{\eta\eta} + (1 + a\mu)u_{\eta\xi} + (1 + a\mu)u_{\xi\eta}\]
    
    此时我们分三种情况:

    $a \neq 0$时,我们取$\mu = -\frac{1}{a} \neq \lambda$,那么方程化为
    \[u_{\xi\xi} + \frac{b}{a^2} u_{\eta\eta} = 0\]

    $a = 0$且$b < 0$时,我们取$\mu = \frac{1}{\sqrt{-b}}$,那么方程化为
    \[u_{\xi\xi} + u_{\xi\eta} + u_{\eta\xi} = 0\]

    $a = 0$且$b \geq 0$时,我们取$\mu = 1$即可得到
    \[u_{\xi\xi} + u_{\xi\eta} + u_{\eta\xi} + (1 + b) u_{\eta\eta}= 0\]
\end{exercise}

\begin{exercise}{22}
    由于线性映射数乘的性质,可知给$x$乘上非零系数$c$不改变$\sup\frac{\|Ax\|}{\|x\|}$的结果,我们不妨限定$\|x\| = 1$,由于
    \[\|Ax\|^2 = x^T A^T A x\]

    我们的$A^T A$是自伴随算子[对称矩阵],从而其可对角化称为$PDP^{-1}$的形式,其中$D = \begin{pmatrix}
        \lambda_1&&\\
        &\ddots&\\
        &&\lambda_m\\
    \end{pmatrix}$.然后作变量代换$y = P^{-1}x$,就有
    \[\max_{\|x\| = 1} x^T A^T A x = \max_{\|y\| = 1} y^T D y\]

    显然当$y$为$e_1$时上式最大,取值为$\lambda_1$,也就是说
    \[\sup\frac{\|Ax\|}{\|x\|} = \max_{\|x\| = 1} \|Ax\| = \sqrt{\lambda_1} = \sigma_1\]

    代入我们的$A$,可以得到
    \[A^T A = \begin{pmatrix}
        10&8&6\\
        8&8&8\\
        6&8&10\\
    \end{pmatrix}\]
    可以求得其特征多项式为
    \[p(\lambda) = -\lambda(16 - \lambda)(4 - \lambda)\]

    故其最大特征值为$16$,从而$A$的最大奇异值为$4$,故$A$的范数为$4$
\end{exercise}

\begin{exercise}{23}
    求梯度可得向量$\begin{pmatrix}
        e^{x + z}\sin y\\
        e^{x + z}\cos y\\
        e^{x + z}\sin y\\
    \end{pmatrix}$
    之后利用向量值函数微分的性质,可得
    \[Df = \begin{pmatrix}
        e^{x + z}\sin y&e^{x + z}\cos y&e^{x + z}\sin y\\
        e^{x + z}\cos y&-e^{x + z}\sin y&e^{x + z}\cos y\\
        e^{x + z}\sin y&e^{x + z}\cos y&e^{x + z}\sin y\\
    \end{pmatrix}\]
\end{exercise}

\begin{exercise}{24}
    设三边为$a,b,c$,不妨假设限制条件为$a + b = 1$,利用角$C$余弦定理
    \[\cos C = \frac{a^2 + b^2 - c^2}{2ab}\]

    从而
    \[S(a,b,c) = \frac{1}{2}ab\sin C = \frac{1}{2}ab \sqrt{1 - \parameter{\frac{a^2 + b^2 - c^2}{2ab}}^2}\]

    我们不妨考虑$2S^2$的极值
    \[2S^2 = a(1 - a)(1 - c^2) - \frac{1}{4}(1 - c^2)^2\]

    于是我们求各偏导数
    \[\frac{\partial 2S^2}{\partial a} = (1 - 2a)(1 - c^2) = 0\]
    \[\frac{\partial 2S^2}{\partial c} = -2a(1 - a)c - 4c^3 + 4c = 0\]

    结合实际意义
    \[\begin{cases}
        0 < a < 1\\
        c > 0\\
    \end{cases}\]

    可解出
    \[\begin{cases}
        a = \frac{1}{2}\\
        b = \frac{1}{2}\\
        c = \frac{\sqrt{14}}{4}\\
    \end{cases}\]

    解出
    \[S = \frac{\sqrt{7}}{32}\]

    接下来我们利用Hessian矩阵来验证这是极大值,我们有
    \[H = \begin{pmatrix}
        2c^2 - 2&4ac - 2c\\
        4ac - 2c&2a^2 - 2a - 12c^2 + 4\\
    \end{pmatrix} = \begin{pmatrix}
        -\frac{1}{4}&0\\
        0&-7\\
    \end{pmatrix}\]

    这个矩阵是负定的,这证明了我们求得的这个临界点是极大值点,故我们求得的$S$是极大值.
\end{exercise}

\begin{exercise}{25}
    我们记$\angle BOC = \alpha,\angle AOC = \beta$,那么面积可以表示为
    \[S = \frac{1}{2}\sin \alpha + \frac{1}{2}\sin \beta + \frac{1}{2} \sin (\pi - \alpha - \beta)\]

    我们来求
    \[2S = \sin \alpha + \sin \beta + \sin(\alpha + \beta)\]

    的极值,从而利用
    \[\frac{\partial 2S}{\partial \alpha} = \cos \alpha + \cos (\alpha + \beta) = 0\]
    \[\frac{\partial 2S}{\partial \beta} =  \cos \beta + \cos (\alpha + \beta) = 0\]

    在
    \[\begin{cases}
        0 < \alpha < \pi\\
        0 < \beta < \pi\\
        0 < \alpha + \beta < \pi\\
    \end{cases}\]

    的条件下,我们可以解出$\alpha = \beta$,进而
    \[\cos \alpha = \frac{1}{2}\]
    \[\alpha = \beta = \frac{\pi}{3}\]

    从而
    \[S = \frac{3\sqrt{3}}{4}\]

    之后我们求出Hessian矩阵
    \[H = \begin{pmatrix}
        -\sin \alpha - \sin(\alpha + \beta)&-\sin(\alpha + \beta)\\
        -\sin(\alpha + \beta)&-\sin \beta - \sin(\alpha + \beta)\\
    \end{pmatrix} = \begin{pmatrix}
        -\sqrt{3}&-\frac{\sqrt{3}}{2}\\
        -\frac{\sqrt{3}}{2}&-\sqrt{3}\\
    \end{pmatrix}\]

    其奇数阶(这里就是一阶)顺序主子式小于$0$,偶数阶顺序主子式(这里就是二阶)大于$0$,故矩阵负定,从而此时取到极大值.
\end{exercise}

\begin{exercise}{26}

    % \[\frac{\partial L}{\partial a} = \frac{\partial L}{\partial f} \frac{\partial f}{\partial a} = (-2y_i + 2(ax_i^2 + bx_i + c)) x_i^2\]
    % \[\frac{\partial L}{\partial b} = \frac{\partial L}{\partial f} \frac{\partial f}{\partial b} = (-2y_i + 2(ax_i^2 + bx_i + c)) x_i\]
    % \[\frac{\partial L}{\partial c} = \frac{\partial L}{\partial f} \frac{\partial f}{\partial c} = (-2y_i + 2(ax_i^2 + bx_i + c))\]

    我们不妨设
    \[f_1(x) = x^2,f_2(x) = x,f_3(x) = x\]

    这样
    \[\begin{aligned}
        L(a,b,c) = \sum_{i = 1}^{n}(y_i - a f_1(x_i) - bf_2(x_i) - c f_3(x_i))^2
    \end{aligned}\]


    我们对于各参数求二阶偏导
    \[\frac{\partial L}{\partial a} = - \sum_{i = 1}^{n} 2(y_i - a f_1(x_i) - bf_2(x_i) - c f_3(x_i)) f_1(x_i) = 0\]
    \[\frac{\partial L}{\partial b} = - \sum_{i = 1}^{n} 2(y_i - a f_1(x_i) - bf_2(x_i) - c f_3(x_i)) f_2(x_i) = 0\]
    \[\frac{\partial L}{\partial c} = - \sum_{i = 1}^{n} 2(y_i - a f_1(x_i) - bf_2(x_i) - c f_3(x_i)) f_3(x_i) = 0\]

    写成矩阵的形式也就是
    \[
    \begin{pmatrix}
        \sum_{i = 1}^{n} f_1^2(x_i)&\sum_{i = 1}^{n} f_1(x_i) f_2(x_i)&\sum_{i = 1}^{n} f_1(x_i) f_3(x_i)\\
        \sum_{i = 1}^{n} f_1(x_i) f_2(x_i)&\sum_{i = 1}^{n} f_2^2(x_i)&\sum_{i = 1}^{n} f_2(x_i) f_3(x_i)\\
        \sum_{i = 1}^{n} f_1(x_i) f_3(x_i)&\sum_{i = 1}^{n} f_2(x_i) f_3(x_i)&\sum_{i = 1}^{n} f_3^2(x_i)\\
    \end{pmatrix}    
    \begin{pmatrix}
        a\\
        b\\
        c\\
    \end{pmatrix}
    =
    \begin{pmatrix}
        \sum_{i = 1}^{n}y_i f_1(x_i)\\
        \sum_{i = 1}^{n}y_i f_2(x_i)\\
        \sum_{i = 1}^{n}y_i f_3(x_i)\\
    \end{pmatrix}\]

    同样我们可以得到Hessian矩阵
    \[H = 2 \begin{pmatrix}
        \sum_{i = 1}^{n} f_1^2(x_i)&\sum_{i = 1}^{n} f_1(x_i) f_2(x_i)&\sum_{i = 1}^{n} f_1(x_i) f_3(x_i)\\
        \sum_{i = 1}^{n} f_1(x_i) f_2(x_i)&\sum_{i = 1}^{n} f_2^2(x_i)&\sum_{i = 1}^{n} f_2(x_i) f_3(x_i)\\
        \sum_{i = 1}^{n} f_1(x_i) f_3(x_i)&\sum_{i = 1}^{n} f_2(x_i) f_3(x_i)&\sum_{i = 1}^{n} f_3^2(x_i)\\
    \end{pmatrix} \]

    % 由于我们这里的函数是特殊的,因此有
    % \[
    % \begin{aligned}
    % \det(\frac{1}{2}H) &= \det\begin{pmatrix}
    %     \sum_{i = 1}^{n}x_i^4&\sum_{i = 1}^{n}x_i^3&\sum_{i = 1}^{n}x_i^2\\
    %     \sum_{i = 1}^{n}x_i^3&\sum_{i = 1}^{n}x_i^2&\sum_{i = 1}^{n}x_i^1\\
    %     \sum_{i = 1}^{n}x_i^2&\sum_{i = 1}^{n}x_i^1&n\\
    % \end{pmatrix}\\
    % &= n \sum_{i = 1}^{n}x_i^4 \sum_{i = 1}^{n}x_i^2 - n \parameter{\sum_{i = 1}^{n}x_i^3}^2 + \sum_{i = 1}^{n}x_i^3 \sum_{i = 1}^{n}x_i^2 \sum_{i = 1}^{n}x_i^1 - \parameter{\sum_{i = 1}^{n}x_i^2}^3 \\
    % &+ \sum_{i = 1}^{n}x_i^3 \sum_{i = 1}^{n}x_i^2 \sum_{i = 1}^{n}x_i^1 - \sum_{i = 1}^{n}x_i^4 \parameter{\sum_{i = 1}^{n}x_i}^2\\
    % &\geq \sum_{i = 1}^{n}x_i^3 \sum_{i = 1}^{n}x_i^2 \sum_{i = 1}^{n}x_i^1 - \sum_{i = 1}^{n}x_i^4 \parameter{\sum_{i = 1}^{n}x_i}^2\\
    % &= \parameter{\sum_{i = 1}^{n}x_i^3 \sum_{i = 1}^{n}x_i^2 - \sum_{i = 1}^{n}x_i^4 \sum_{i = 1}^{n}x_i}\sum_{i = 1}^{n}x_i\\
    % &= \parameter{\sum_{1\leq i , j \leq n, i \neq j}(x_i^3 x_j^2 - x_j^4 x_i)}\sum_{i = 1}^{n}x_i\\
    % &\geq 0
    % \end{aligned}\]

    % 第一第二项,第三第四项分别运用柯西不等式可证明最终结果$\geq 0$,由于取等当且仅当所有$x_i$都相等,这在现实问题中不存在,因此可以认为行列式大于$0$.于是我们可以解出
    这个矩阵从多元微积分的角度很难处理,但如果我们考虑从线性代数的角度,我们定义向量
    \[\alpha_1 = \begin{pmatrix}
        f_1(x_1)\\
        \vdots\\
        f_1(x_n)\\
    \end{pmatrix},\alpha_2 = \begin{pmatrix}
        f_2(x_1)\\
        \vdots\\
        f_2(x_n)\\
    \end{pmatrix},\alpha_3 = \begin{pmatrix}
        f_3(x_1)\\
        \vdots\\
        f_3(x_n)\\
    \end{pmatrix}\]

    那么
    \[\frac{1}{2} H = \begin{pmatrix}
        \alpha_1^T\\
        \alpha_2^T\\
        \alpha_3^T\\
    \end{pmatrix}
    \begin{pmatrix}
        \alpha_1&\alpha_2&\alpha_3\\
    \end{pmatrix} = A^T A\]
    我们利用线性代数的性质不难有对于任意的$v \in \RR^3,v \neq 0$,有
    \[\frac{1}{2} v^T H v = v^T A^T A v = \|A v\|^2 \geq 0\]
    
    故$H$是半正定的,对于取等条件也就是存在$v \in \RR^3$使得$\alpha_1 v_1 + \alpha_2 v_2 + \alpha_3 v_3 = 0$,也就是$\alpha_1,\alpha_2,\alpha_3$线性相关.然而在现实问题中,我们的三个$\RR^n$中的向量共面需要满足非常严苛的条件,基本不可能发生(尤其是$n$足够大时).更为严谨地说,我们此时三个向量分别为
    \[\begin{pmatrix}
        1\\
        \vdots\\
        1\\
    \end{pmatrix},\begin{pmatrix}
        x_1\\
        \vdots\\
        x_n\\
    \end{pmatrix},\begin{pmatrix}
        x_1^2\\
        \vdots\\
        x_n^2\\
    \end{pmatrix}\]
    
    而我们要让等号不成立,也就是$\n A = \{0\}$,也即$\rank A = 3$,其充分必要条件为存在$A$的一个不为零的三阶子式,形如
    \[\det\begin{pmatrix}
        1&x_i&x_i^2\\
        1&x_j&x_j^2\\
        1&x_k&x_k^2\\
    \end{pmatrix} = (x_k - x_j)(x_j - x_i)(x_k - x_i)\]
    
    (这是一个范德蒙行列式),而因此只要存在三个互不相同的$x_i$,就足以保证这个矩阵$H$正定!
    
    因此可以认为在现实问题中,矩阵$H$是正定的,因此我们就有:
    
    \[\begin{pmatrix}
        a\\
        b\\
        c\\
    \end{pmatrix}
    =\begin{pmatrix}
        \sum_{i = 1}^{n}x_i^4&\sum_{i = 1}^{n}x_i^3&\sum_{i = 1}^{n}x_i^2\\
        \sum_{i = 1}^{n}x_i^3&\sum_{i = 1}^{n}x_i^2&\sum_{i = 1}^{n}x_i^1\\
        \sum_{i = 1}^{n}x_i^2&\sum_{i = 1}^{n}x_i^1&n\\
    \end{pmatrix}^{-1}
    \begin{pmatrix}
        \sum_{i = 1}^{n}y_i x_i^2\\
        \sum_{i = 1}^{n}y_i x_i\\
        \sum_{i = 1}^{n}y_i \\
    \end{pmatrix}
    \]

    如果一定要解出来,那么有
     \[
     a = \frac{\sum_{i=1}^{n} x_i^2 y_i - (\sum_{i=1}^{n} x_i^2)(\sum_{i=1}^{n} y_i)}{\sum_{i=1}^{n} x_i^4 - (\sum_{i=1}^{n} x_i^2)^2}
     \]


     \[
     b = \frac{\sum_{i=1}^{n} x_i y_i - (\sum_{i=1}^{n} x_i)(\sum_{i=1}^{n} y_i)}{\sum_{i=1}^{n} x_i^2 - (\sum_{i=1}^{n} x_i)^2}
     \]

     \[
     c = \frac{\sum_{i=1}^{n} y_i - a(\sum_{i=1}^{n} x_i^2) - b(\sum_{i=1}^{n} x_i)}{n}
     \]

     其中$c$的表达式过于复杂,我们没有代入$a,b$的解析形式

    与此同时我们知道了$H$正定,从而可以在所取得$a,b,c$时取得极小值.
\end{exercise}


\end{CJK}
\end{document}

\begin{itemize} 
\end{itemize}