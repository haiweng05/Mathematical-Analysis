\documentclass[11pt,hyperref,a4paper,UTF8]{ctexart}
\usepackage[left=2.50cm, right=2.50cm, top=2.50cm, bottom=2.50cm]{geometry}
\usepackage[unicode=true,colorlinks,urlcolor=blue,linkcolor=blue,bookmarksnumbered=true]{hyperref}
\usepackage{graphicx} % Required for inserting images
\usepackage{ctex}

\usepackage{amsfonts}
\usepackage{amssymb}
\usepackage{amsmath}
\usepackage{amsthm}
% 导入 xparse 宏包以支持 LaTeX3 语法
\usepackage{xparse}
\usepackage{pgfplots}
% 用于插入带有坐标轴、标签和曲线的图像

% \begin{tikzpicture}
%     \begin{axis}[
%       xlabel=$x$,
%       ylabel=$f(x)$,
%       axis lines=middle,
%       xmin=-5, xmax=5,
%       ymin=-2, ymax=8,
%       width=\textwidth,
%       height=8cm
%     ]
%     \addplot[blue,domain=-3:3] {x^2};
%     \end{axis}
%   \end{tikzpicture}
\usepackage{tikz}
% 用于绘制一般的图像

% \begin{tikzpicture}[scale=0.8]
%     \draw[->] (-4,0) -- (4,0) node[right] {$x$};
%     \draw[->] (0,-1) -- (0,9) node[above] {$f(x)$};
%     \draw[domain=-3:3,smooth,variable=\x,blue] plot ({\x},{\x^2});
%   \end{tikzpicture}

\newtheorem{theorem}{Theorem}[subsection]
\newtheorem{lemma}{Lemma}[subsection]
\newtheorem{corollary}{corollary}[subsection]
\newtheorem{example}{Example}[subsection]
\newtheorem{definition}{Definition}[subsection]
% 为了证明中可以使用中文,后续定义证明时使用cproof而不是proof
\newenvironment{cproof}{%
{
    \textbf{Proof\\}
    }
}{
%   \hfill $\square$ 添加结束符号
%   \par\bigskip 可选的垂直间距
}
\newenvironment{exercise}[1]{%
{\textbf{Exercise #1} \\ 
    }
}{
  \hfill $\square$ 
  \par\bigskip 
}

\newenvironment{solution}{%
{
    \textbf{Solution\\}
    }
}{
  \hfill $\square$ 
  \par\bigskip 
}


% \newenvironment{identification}{%
% \heiti{定义}\kaishu
% }{%
% %   \hfill $\square$ 添加结束符号
% %   \par\bigskip 可选的垂直间距
% }

\newcommand{\RR}{\mathbb{R}}
\newcommand{\NN}{\mathbb{N}}
\newcommand{\CC}{\mathbb{C}}
\newcommand{\QQ}{\mathbb{Q}}
\newcommand{\ZZ}{\mathbb{Z}}
\newcommand{\FF}{\mathbb{F}}
\newcommand{\PP}{\mathbb{P}}
% 简化各种常见数的集合

\newcommand{\parameter}[1]{\left(#1\right)}

\newcommand{\bracket}[1]{\left[#1\right]}

\newcommand{\abs}[1]{\left|#1\right|}
% 各种自动变化大小的括号的简化

\newcommand{\ve}{\boldsymbol}
% 为了适应David C Lay线性代数中,简化斜体+粗体向量的书写

\newcommand{\base}{\mathcal}

\newcommand{\tb}{\textbf}

\newcommand{\col}{\text{Col}}

\newcommand{\row}{\text{Row}}
\newcommand{\nul}{\text{Nul}}
\newcommand{\spans}{\text{Span}}
\newcommand{\proj}{\text{proj}}
\newcommand{\adj}{\text{adj.}}
\newcommand{\rank}{\text{rank}}
\newcommand{\range}{\text{range}}
\newcommand{\n}{\text{null}}
\newcommand{\tr}{\text{tr}}
\newcommand{\sign}{\text{sign}}
\newcommand{\perm}{\text{perm}}
% 简化粗体字体的书写

\newcommand{\f}[2]{\frac}
\newcommand{\df}[2]{\dfrac}

\newcommand{\ip}[1]{\left<#1\right>}

\newcommand{\pa}{\paragraph}
\newcommand{\spa}{\subparagraph}
\newcommand{\se}{\section}
\newcommand{\sse}{\subsection}
\newcommand{\ssse}{\subsubsection}

% \NewDocumentCommand{\vs}{m m m}{
%     \ve{#1}_{#2},\cdots,\ve{#1}_{#3}
% }
% % 快速书写一个向量组,第一个参数为向量名称,后两个为首末角标

% % $\vs{b}{1}{n} $这是多参数命令的使用示例

% \NewDocumentCommand{\cvs}{m m m m m}{
%     #1_{#4}\ve{#2}_{#4} #3 \cdots #3 #1_{#5}\ve{#2}_{#5}
% }
% % 快速书写一个向量线性组合,第一个参数为系数,第二个参数为向量名称,第三个参数为运算符,后两个参数为角标

\NewDocumentCommand{\size}{m m}{
    #1\times#2
}


\title{习题十二}
\author{徐海翁}
\date{2024.5.22}

\begin{document}

\begin{CJK}{UTF8}{gkai}

\maketitle
% \tableofcontents
\begin{exercise}{9}
    若无特殊说明,本题所有小问均采取极坐标变换
    \[\begin{cases}
        x = r\cos \theta\\
        y = r\sin \theta\\
    \end{cases}\]

    我们可以求得这个变换的雅可比行列式为$r$.\\

    (1)
    \[
    \begin{aligned}
    \iint_{1 \leq x^2 + y^2 \leq 9} \sin \sqrt{x^2 + y^2}\, d(x,y) &= \int_{[1,3]\times[0,2\pi]} r\sin r \, d(r,\theta)\\
    &= \int_{0}^{2\pi}\, d\theta \int_{1}^{3} r\sin r\, dr\\
    &= 2\pi \cdot (-3\cos 3 + \sin 3 + 3\cos 1 - \sin 1)\\
    \end{aligned}
    \]

    (2)这里改用变量代换
    \[\begin{cases}
        x = 1 + r\cos \theta\\
        y = 1 + r\sin \theta\\
    \end{cases}\]

    可以发现雅可比行列式仍然为$r$,于是
    \[
    \begin{aligned}
    \iint_{x^2 + y^2 \leq 2x + 2y}(3x + 2y)\, d(x,t)&= \int_{[0,\sqrt{2}]\times[0,2\pi]} (5r + r^2(3\cos \theta + 2\sin\theta)) \, d(r,\theta)\\
    &= \int_{0}^{\sqrt{2}}\, dr \int_{0}^{2\pi} (5r + r^2(3\cos \theta + 2\sin\theta))\, d\theta\\
    &= \int_{0}^{\sqrt{2}}\, dr 10\pi r\\
    &= 10\pi\\
    \end{aligned}  
    \]
    (3)
    \[\begin{aligned}
        \iint_{x^2 + y^2 \leq a^2} \tan(x^2 + y^2)\,d(x,y) &= \int_{[0,a]\times[0,2\pi]} \tan(r^2) r d(r,\theta)\\
        &= \int_{0}^{2\pi}\,d\theta \int_{0}^{a}\tan(r^2) \, d(r^2)\\
        &= 2\pi \cdot (-\frac{1}{2}\ln(\cos r^2))\bigg|_0^a\\
        &= -\pi \ln(\cos a^2)\\
    \end{aligned}\]

    (4)\[\begin{aligned}
        \iint_{x^2 + y^2 < 1} \frac{1}{\sqrt{1 - x^2 - y^2}}\,d(x,y) &= \lim_{\delta \to 0+}\int_{[0,1 - \delta]\times[0,2\pi]} \frac{r d(r,\theta)}{\sqrt{1 - r^2}}\\
        &= \int_{0}^{2\pi} \, d\theta \lim_{\delta \to 0+}\int_{0}^{1 - \delta} \frac{r dr}{\sqrt{1 - r^2}}\\
        &= \int_{0}^{2\pi} \, d\theta \lim_{\delta \to 0+}(-\sqrt{1 - r^2})\bigg|_0^{1 - \delta}\\
        &= 2\pi\\
    \end{aligned}\]
\end{exercise}

\begin{exercise}{10}
    (1)利用对称性可知
    \[
    \begin{aligned}
        \iiint_V(x^3 y^2 z + xy^3z^2 + x^2 y z^3)\, d(x,y,z) &= \iiint_V x^2 y z^3\, d(x,y,z)\\
        &= \int_{-1}^{1}\, dx \int_{0}^{\sqrt{1 - x^2}}\, dy\int_{0}^{\sqrt{1 - x^2 - y^2}}x^2 y z^3\, dz\\
        &= \int_{-1}^{1}\, dx \int_{0}^{\sqrt{1 - x^2}}\, dy \frac{1}{4}x^2 y(1 - x^2 - y^2)^2\\
        &= \int_{-1}^{1} \frac{1}{24}x^2(1 - x^2)^3\\
        &= \frac{4}{945}\\
    \end{aligned}    
    \]
    (2)做变量替换
    \[\begin{cases}
        x = r \cos \theta\\
        y = r \sin \theta\\
        z = z\\
      \end{cases}\]

      此时得到雅可比行列式为$r$,积分转化为
      \[
      \begin{aligned}
        \int_{0}^{2\pi} & d\theta\int_{0}^{\sqrt{\frac{\sqrt{5} - 1}{2}}}\,dr \int_{r^2}^{\sqrt{1 - r^2}}r r^3(\sin^3 \theta+ \cos^3\theta) + r z^3\, dz \\
        &=\int_{0}^{2\pi}d\theta\int_{0}^{\sqrt{\frac{\sqrt{5} - 1}{2}}}\,dr \int_{r^2}^{\sqrt{1 - r^2}}r^4(\sin^3 \theta+ \cos^3\theta) + \int_{0}^{2\pi}d\theta\int_{0}^{\sqrt{\frac{\sqrt{5} - 1}{2}}}\,dr \int_{r^2}^{\sqrt{1 - r^2}} rz^3\, dz\\
        &= \int_{0}^{2\pi}d\theta\int_{0}^{\sqrt{\frac{\sqrt{5} - 1}{2}}}\,dr\int_{r^2}^{\sqrt{1 - r^2}} rz^3\, dz\\
        &= \int_{0}^{2\pi}d\theta\int_{0}^{\sqrt{\frac{\sqrt{5} - 1}{2}}}\,r dr \parameter{\frac{1}{4}z^4}\bigg|_{r^2}^{\sqrt{1 - r^2}} \\  
        &= \frac{25\sqrt{5} - 47}{120}\pi\\ 
    \end{aligned}  
      \]

    (3)我们做球坐标变换
    \[\begin{cases}
        x = r \cos \varphi \cos \theta\\
        y = r \cos \varphi \sin \theta\\
        z = r \sin \varphi\\
      \end{cases}\]

      这个变换的雅可比行列式为$r^2 \cos \varphi$,于是原积分转化为
      \[
      \begin{aligned}
        \lim_{a \to +\infty}\int_{[1,a] \times [0,2\pi] \times[0,\pi]} \frac{r^2 \cos \varphi}{r^4}\, d(r,\theta,\varphi) &= \int_{0}^{2\pi}\, d\theta \int_{-\frac{1}{2}\pi}^{\frac{1}{2}\pi}\cos \varphi \, d\varphi \lim_{a \to +\infty}\int_{1}^{a} \frac{1}{r^2}\, dr\\
        &= 4\pi\\
      \end{aligned}  
      \]
\end{exercise}

\begin{exercise}{11}
    (1)转换为累次积分
    \[\int_{0}^{1}\,dz \int_{0}^{6\sqrt{1 - z^2}}\, dy \int_{0}^{4\sqrt{1 - z^2} - \frac{2}{3}y}\, dx = \int_{0}^{1}\,dz \int_{0}^{6\sqrt{1 - z^2}}\, dy \parameter{4\sqrt{1 - z^2} - \frac{2}{3}y} = 8\]

    (2)转换为累次积分
    \[\int_{0}^{2} dy \int_{-y}^{1 - y} (x + y)\, dx = 1\]
\end{exercise}

\begin{exercise}{12}
    (1)
    \[
    \begin{aligned}    
        \int_{\Gamma} (2x + 3y) ds &= \int_{AB} (2x + 3y) ds + \int_{BC} (2x + 3y) ds + \int_{CD} (2x + 3y) ds + \int_{DA} (2x + 3y) ds\\
        &= \int_{0}^2 (0 + 3y) dy + \int_{0}^3 (2x + 6) dx + \int_{2}^0 (6 + 3y) d(-y) + \int_{3}^0 (2x + 0) d(-x)\\
        &= 60\\
    \end{aligned}    
    \]

    (2)记这里的螺旋线为$r$,则
    \[r'(t) = (-2\sin t,2 \cos t, 1)\]

    于是
    \[\|r'(t)\| = \sqrt{5}\]

    \[\int_{\Gamma} (x^2 + y^2 + 2z^2)\, ds = \int_{0}^{3\pi} (4 + 2t^2) \sqrt{5}\, dt = 12\sqrt{5}\pi + 18\sqrt{5}\pi^3\]

    (3)利用对称性,我们考虑第一象限内的部分,参数化这部分曲线为
    \[r(\theta) = (a\cos \theta,b\sin \theta)\]

    从而有
    \[r'(\theta) = (-a\sin \theta,b\cos\theta)\]
    \[\|r'(\theta)\| = \sqrt{a^2 \sin^2 \theta + b^2 \cos^2 \theta}\]

    于是我们的积分就表示为
    \[
    \begin{aligned}
        4\int_{0}^{\frac{\pi}{2}} \frac{\sqrt{a^2 \sin^2 \theta + b^2 \cos^2 \theta}}{a^2 \sin^\theta + b^2 \cos^2 \theta}d\theta &= 4\int_{0}^{\frac{\pi}{2}} \frac{\sqrt{b^2 - (b^2 - a^2)\sin^2 \theta}}{a^2  - (a^2 - b^2) \sin^2\theta}d\theta\\
        &= \frac{4b}{a^2}\int_{0}^{\frac{\pi}{2}}\frac{\sqrt{1 - \frac{b^2 - a^2}{b^2} \sin^2 \theta}}{1 - \frac{a^2 - b^2}{a^2}\sin^2\theta}\, d\theta\\
        &= \frac{4b}{a^2}\int_{0}^{1}\frac{1 - \frac{b^2 - a^2}{b^2} x^2}{\parameter{\sqrt{1 - x^2}}\parameter{\sqrt{1 - \frac{b^2 - a^2}{b^2} x^2}}\parameter{1 - \frac{a^2 - b^2}{a^2}x^2}}\, dx\\
        &= \frac{4}{b}\int_{0}^{1} \frac{1}{\parameter{\sqrt{1 - x^2}}\parameter{\sqrt{1 - \frac{b^2 - a^2}{b^2} x^2}}}\, dx\\ &+ \frac{4b}{a^2}\int_{0}^{1}\frac{1 - \frac{a^2}{b^2}}{\parameter{\sqrt{1 - x^2}}\parameter{\sqrt{1 - \frac{b^2 - a^2}{b^2} x^2}}\parameter{1 - \frac{a^2 - b^2}{a^2}x^2}} \, dx\\
    \end{aligned}    
    \]
    
    这是第二型椭圆积分与第三型椭圆积分的和,对于$a,b$没有解析形式的积分值.
\end{exercise}

\begin{exercise}{13}
    (1)设
    \[\begin{cases}
        x = \frac{a}{2} + \frac{a}{2} \cos \theta\\
        y = \frac{a}{2} \sin \theta\\
        z = z\\
    \end{cases}\]

    % 其中$\cos \theta$满足约束条件
    % \[\cos \theta \leq 7 - 2 \frac{z^2}{a^2}\] 

    % 故
    % \[\frac{z^2}{a^2}\geq 3\]
    % 时,$\theta$的取值范围受限为
    % \[\bracket{\arccos\parameter{7 - 2\frac{z^2}{a^2}},2\pi - \arccos\parameter{7 - 2\frac{z^2}{a^2}}}\]
    其中$z$满足约束
    \[- a \sqrt{\frac{7}{2}- \frac{1}{2}\cos\theta } \leq z \leq a \sqrt{\frac{7}{2}- \frac{1}{2}\cos\theta}\]

    由于我们
    \[r_\theta \times r_z = (\frac{a}{2}\cos \theta,0,0)\]

    利用对称性,原积分转化为积分
    \[
    \begin{aligned}
        \int_D \frac{a}{2}|\cos \theta| d(\theta,z)
        %  &= \int_{0}^{2\pi} \, d\theta \int_{0}^{\sqrt{3}a} \frac{a}{2}|\cos\theta| \, dz +  \int_{\sqrt{3}a}^{2a}  \, dz\int_{\arccos\parameter{7 - 2\frac{z^2}{a^2}}}^{2\pi - \arccos\parameter{7 - 2\frac{z^2}{a^2}}}  \, d\theta\frac{a}{2}|\cos\theta|\\
        &=\int_{0}^{2\pi}\, d\theta \int_{- a \sqrt{\frac{7}{2} - \frac{1}{2}\cos \theta}}^{a \sqrt{\frac{7}{2} - \frac{1}{2}\cos \theta}} \frac{a}{2}|\cos \theta| \, dz\\
        &= a^2 \int_{0}^{\pi} |\cos \theta| \sqrt{14 - 2\cos \theta} d\theta\\ 
        % &= a^2 \int_{0}^{\frac{1}{2}\pi} \cos \theta \sqrt{14 - 2\cos \theta} d\theta - a^2 \int_{\frac{1}{2}\pi}^{\pi} \cos \theta \sqrt{14 - 2\cos \theta} d\theta\\ 
    \end{aligned}    
    \]

    推测这个函数也可以转化为椭圆积分的形式.

    (2)
    利用对称性原积分可化为
    \[4\int_{[0,1]\times[0,1]}u du dv + 1 \int_{[0,1]\times[0,1]} 1 du dv = 3\]

    (3)设
    \[\begin{cases}
        x = a\cos \theta\\
        y = a\sin \theta\\
        z = a^2\\
    \end{cases}\]

    其中$a$满足约束条件$0\leq a \leq 1$
    \[r_\theta\ = (-a\sin\theta,a\cos\theta,0),r_a = (\cos\theta,\sin\theta,2a)\]
    \[r_\theta \times r_a = (2a^2 \cos\theta,-2a^2 \sin\theta,-a)\]
    \[\|r_\theta \times r_a\| = a\sqrt{4a^2 + 1} \]

    于是原积分化为
    \[\int_{0}^{2\pi}\,d\theta \int_{0}^{1} a^2\sqrt{4a^2 + 1} \sin \theta\,da = 0\]
\end{exercise}
\end{CJK}
\end{document}

\begin{itemize} 
\end{itemize}