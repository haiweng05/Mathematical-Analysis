\documentclass[11pt,hyperref,a4paper,UTF8]{ctexart}
\usepackage[left=2.50cm, right=2.50cm, top=2.50cm, bottom=2.50cm]{geometry}
\usepackage[unicode=true,colorlinks,urlcolor=blue,linkcolor=blue,bookmarksnumbered=true]{hyperref}
\usepackage{graphicx} % Required for inserting images
\usepackage{ctex}

\usepackage{amsfonts}
\usepackage{amssymb}
\usepackage{amsmath}
\usepackage{amsthm}
% 导入 xparse 宏包以支持 LaTeX3 语法
\usepackage{xparse}
\usepackage{pgfplots}
% 用于插入带有坐标轴、标签和曲线的图像

% \begin{tikzpicture}
%     \begin{axis}[
%       xlabel=$x$,
%       ylabel=$f(x)$,
%       axis lines=middle,
%       xmin=-5, xmax=5,
%       ymin=-2, ymax=8,
%       width=\textwidth,
%       height=8cm
%     ]
%     \addplot[blue,domain=-3:3] {x^2};
%     \end{axis}
%   \end{tikzpicture}
\usepackage{tikz}
% 用于绘制一般的图像

% \begin{tikzpicture}[scale=0.8]
%     \draw[->] (-4,0) -- (4,0) node[right] {$x$};
%     \draw[->] (0,-1) -- (0,9) node[above] {$f(x)$};
%     \draw[domain=-3:3,smooth,variable=\x,blue] plot ({\x},{\x^2});
%   \end{tikzpicture}

\newtheorem{theorem}{Theorem}[subsection]
\newtheorem{lemma}{Lemma}[subsection]
\newtheorem{corollary}{corollary}[subsection]
\newtheorem{example}{Example}[subsection]
\newtheorem{definition}{Definition}[subsection]
% 为了证明中可以使用中文,后续定义证明时使用cproof而不是proof
\newenvironment{cproof}{%
{
    \textbf{Proof\\}
    }
}{
%   \hfill $\square$ 添加结束符号
%   \par\bigskip 可选的垂直间距
}
\newenvironment{exercise}[1]{%
{\textbf{Exercise #1} \\ 
    }
}{
  \hfill $\square$ 
  \par\bigskip 
}

\newenvironment{solution}{%
{
    \textbf{Solution\\}
    }
}{
  \hfill $\square$ 
  \par\bigskip 
}


% \newenvironment{identification}{%
% \heiti{定义}\kaishu
% }{%
% %   \hfill $\square$ 添加结束符号
% %   \par\bigskip 可选的垂直间距
% }

\newcommand{\RR}{\mathbb{R}}
\newcommand{\NN}{\mathbb{N}}
\newcommand{\CC}{\mathbb{C}}
\newcommand{\QQ}{\mathbb{Q}}
\newcommand{\ZZ}{\mathbb{Z}}
\newcommand{\FF}{\mathbb{F}}
\newcommand{\PP}{\mathbb{P}}
% 简化各种常见数的集合

\newcommand{\parameter}[1]{\left(#1\right)}

\newcommand{\bracket}[1]{\left[#1\right]}

\newcommand{\abs}[1]{\left|#1\right|}
% 各种自动变化大小的括号的简化

\newcommand{\ve}{\boldsymbol}
% 为了适应David C Lay线性代数中,简化斜体+粗体向量的书写

\newcommand{\base}{\mathcal}

\newcommand{\tb}{\textbf}

\newcommand{\col}{\text{Col}}

\newcommand{\row}{\text{Row}}
\newcommand{\nul}{\text{Nul}}
\newcommand{\spans}{\text{Span}}
\newcommand{\proj}{\text{proj}}
\newcommand{\adj}{\text{adj.}}
\newcommand{\rank}{\text{rank}}
\newcommand{\range}{\text{range}}
\newcommand{\n}{\text{null}}
\newcommand{\tr}{\text{tr}}
\newcommand{\sign}{\text{sign}}
\newcommand{\perm}{\text{perm}}
% 简化粗体字体的书写

\newcommand{\f}[2]{\frac}
\newcommand{\df}[2]{\dfrac}

\newcommand{\ip}[1]{\left<#1\right>}

\newcommand{\pa}{\paragraph}
\newcommand{\spa}{\subparagraph}
\newcommand{\se}{\section}
\newcommand{\sse}{\subsection}
\newcommand{\ssse}{\subsubsection}

% \NewDocumentCommand{\vs}{m m m}{
%     \ve{#1}_{#2},\cdots,\ve{#1}_{#3}
% }
% % 快速书写一个向量组,第一个参数为向量名称,后两个为首末角标

% % $\vs{b}{1}{n} $这是多参数命令的使用示例

% \NewDocumentCommand{\cvs}{m m m m m}{
%     #1_{#4}\ve{#2}_{#4} #3 \cdots #3 #1_{#5}\ve{#2}_{#5}
% }
% % 快速书写一个向量线性组合,第一个参数为系数,第二个参数为向量名称,第三个参数为运算符,后两个参数为角标

\NewDocumentCommand{\size}{m m}{
    #1\times#2
}


\title{习题十}
\author{徐海翁}
\date{2024.5.8}

\begin{document}

\begin{CJK}{UTF8}{gkai}

\maketitle
% \tableofcontents
\begin{exercise}{33}
    由点$(a,0,0)$和方向$(dy,dz) \parallel (1,1)$可以定义一个截面
    \[y - z = 0\]

    由于这里的方向恰好定义了$(a,0,0)$处的切向量$e$,且曲面在该点的法线$n = (1,0,0)$也在这个平面内,故这个平面与曲面的交线即为我们的法截线,即有法截线为
    \[
    \begin{cases}
        \frac{x^2}{a^2} + \frac{y^2}{b^2} + \frac{z^2}{c^2}= 1\\
        z = y\\
    \end{cases}    
\]
    我们将这个法截线视作关于$t$的曲线
    \[r(t) = (x(t),y(t),y(t))\]
    在法截线方程两侧对$t$求导,得到
    \[r' = (x'(t),y'(t),y'(t))\]
    \[r'' = (x''(t),y''(t),y''(t))\]

    利用前面的结论可得
    \[k_n = \frac{\|r' \times r''\|}{\|r'\|^3}\]
    \[r' \times r'' = (0,y' x'' - x' y'' ,x' y'' - y' x'')\]

    在$(a,0,0)$处对$t$求偏导数有
    \[
    \begin{cases}
        x'(t) = 0\\
        \frac{x''}{a} + \frac{b^2 + c^2}{b^2 c^2}(y')^2 = 0\\
    \end{cases}    
    \]
    \[k_n = \frac{\sqrt{2} |y' x''|}{(\sqrt{2}|y'|)^3} = \frac{1}{2} \frac{|x''|}{|y'|^2} = \frac{a(b^2 + c^2)}{2b^2 c^2}\]

    \[\rho_n = \frac{1}{k_n} = \frac{2b^2 c^2}{a(b^2 + c^2)}\]
\end{exercise}

\begin{exercise}{34}
    考虑把$\theta,\varphi$视作关于$t$的函数,那么有
    \[r_\theta = (- \sin \theta \cos \varphi, - \sin \theta \sin \varphi, \cos \theta)\]
    \[r_\varphi = (- \cos \theta \sin \varphi, \cos \theta \cos \varphi,0)\]

    从而就有
    \[E = r_\theta \cdot r_\theta = 1\]
    \[F = r_\varphi \cdot r_\theta = 0\]
    \[G = r_\varphi \cdot r_\varphi = \cos^2 \theta\]

    于是曲面的第一标准形式是
    \[I(d\theta,d\varphi) =  \,d\theta^2 +\cos^2 \theta \, d \varphi^2\]

    我们在一阶导数的基础上得到二阶偏导数
    \[r_{\theta\theta} = (-\cos \theta \cos \varphi, - \cos \theta \sin \varphi,- \sin \theta)\]
    \[r_{\theta\varphi} = (\sin \theta \sin \varphi, - \sin \theta \cos \varphi, 0)\]
    \[r_{\varphi\varphi} = (-\cos \theta \cos \varphi, - \cos \theta \sin \varphi, 0)\]

    同时利用叉乘可以得到
    \[r_\theta \times r_\varphi = (-\cos^2 \theta \cos \varphi, - \cos^2 \theta \sin \varphi, - \sin \theta \cos \theta)\]

    同时我们定义法向量
    \[n = \frac{r_\theta \times r_\varphi}{\|r_\theta \times r_\varphi\|} = (-\cos \varphi, - \sin\varphi, - \tan \theta)\]
    
    从而我们得到
    \[L = n \cdot r_{\theta\theta} = \cos \theta + \sin \theta \tan \theta = \sec \theta\]
    \[M = n \cdot r_{\theta\varphi} = 0\]
    \[N = n \cdot r_{\varphi\varphi} = \cos \theta\]

    于是曲面的第二标准形式是
    \[II(d\theta,d\varphi) = \sec \theta\,d\theta^2 + \cos \theta\, d \varphi^2\]
\end{exercise}
\begin{exercise}{1}
% 对$[a,b]$和$[c,d]$分别做分割$a = x_0 < x_1 < \cdots < x_n = b$, $c = y_0 < y_1 < \cdots < y_m = d$,这样形成了$mn$个闭方块$J_{ij}$,我们可以使其满足$|J_{ij}| < \delta$.

% 不妨设$g(y)$在$[c,d]$上单调递增,那么我们有
% \[\omega_{ij} = \sup_{(x,y),(x',y')}\left|f(x)g(y) - f(x') g(y')\right|\]

利用一元函数的可积性定理,我们可知$f(x),g(y)$均是可积的,我们接下来在闭方块$Q$上考察函数$\tilde{f}(x,y) = f(x),\tilde{g}(x,y) = g(y)$.由于前面的可积条件,我们知道,$\forall \varepsilon > 0$,$\exists \delta > 0$满足$|P| < \delta$时,就有
\[\Omega(f) < \frac{1}{d - c}\varepsilon\]

我们首先考察$\tilde{f}(x,y)$:

对$[a,b]$和$[c,d]$分别做分割$a = x_0 < x_1 < \cdots < x_n = b$, $c = y_0 < y_1 < \cdots < y_m = d$,这样形成了$mn$个闭方块$J_{ij}$,我们可以使其满足$|J_{ij}| < \delta$.

\[\omega_{ij} = \sup_{(x,y),(x',y') \in (\Delta x_i \times \Delta y_j)}\left|\tilde{f}(x,y) - \tilde{f}(x',y')\right| = \sup_{x,x' \in \Delta x_i}\left| f(x) - f(x')\right| = \omega_i\]

从而
\[\Omega(\tilde{f}) = \sum_{J} \omega_{ij}\Delta x_{ij} = \sum_{j} \sum_{i} \omega_{ij}\Delta x_{i} \Delta y_{j} =  \sum_{j} \Omega(f)\Delta y_{j} < \frac{1}{d - c}\varepsilon \cdot (d - c) = \varepsilon\]

同理

\[\Omega(\tilde{g})  < \varepsilon\]

那么也就有$\tilde{f},\tilde{g}$在$Q$上可积,利用书本上的定理可知$f(x)g(y) = \tilde{f}(x,y) \tilde{g}(x,y)$也在$Q$上可积,证毕.
\end{exercise}

\begin{exercise}{2}
(1)由于函数$f(x,y) = e^{x + y}\sin x\cos y$视作关于$y$的函数时在$[c,d]$上可积,故而有
\[
\begin{aligned} 
\iint_{[a,b]\times[c,d] }e^{x + y}\sin x\cos y\, d(x,y) &= \int_{a}^{b}\parameter{e^{x + y}\sin x\cos y \,dy} dx \\
&= \int_{a}^{b} \frac{e^{x + d} (\sin d + \cos d) - e^{x + c} (\sin c + \cos c)}{2} \sin x\, dx\\
&=  \frac{e^{d} (\sin d + \cos d) - e^{c} (\sin c + \cos c)}{2}\int_{a}^{b} e^x \sin x\, dx\\
&= \frac{e^{d} (\sin d + \cos d) - e^{c} (\sin c + \cos c)}{2} \frac{e^{b} (\sin b - \cos b) - e^{a} (\sin a - \cos a)}{2}
\end{aligned} \]

(2)首先函数$f(x,y,z) = x^2 \sin(x + y) (z^2 - x - y)$视作关于$z$的函数时在$[2,3]$可积,从而有
\[\begin{aligned}
    \iint_{[0,1]\times[1,2]\times[2,3]}x^2 \sin(x + y) (z^2 - x - y)\,d(x,y,z) &= \iint_{[0,1]\times[1,2]} x^2 \sin(x + y) \parameter{\frac{19}{3} - (x + y)}\, d(x,y)\\
\end{aligned}\]

之后再将$g(x,y) = x^2 \sin(x + y) \parameter{\frac{19}{3} - (x + y)}$视作关于$y$的函数时在$[1,2]$上可积,因此有
\[\int_{1}^{2} x^2 \sin(x + y) \parameter{\frac{19}{3} - (x + y)} dy = (x^3 - \frac{13}{3}x^2) \cos(x + 2) - (x^3 - \frac{16}{3}x^2) \cos (x + 1) - x^2 \sin (x + 2) + x^2 \sin(x + 1) \]

最终解得原积分为
\[-\frac{38}{3}\cos 3 - \frac{8}{3}\sin 3 + \frac{68}{3} \cos 2 - 7\sin 2 - 8\cos 1 + \frac{32}{3} \sin 1\]
\end{exercise}

(3)函数$f(x,y) = y\tan (x + y^2)$视作关于$y$的函数时在区间$[0,0.5]$上可积
\[\int_{0}^{0.5} y\tan (x + y^2) \, dy = \parameter{-\frac{1}{2}\ln\cos(x + y^2)}\bigg|_{0}^{0.5}= - \frac{1}{2} \ln \cos(x + 0.25) + \frac{1}{2}\ln\cos x\]

之后我们的再对$x$积分,但这个函数无法得到初等的原函数,同理我们如果先对$x$积分,同样可以得到类似的形式,但是再对$y$积分时无法得到表达式形式的结果,因此我们不再求解.
\end{CJK}
\end{document}

\begin{itemize} 
\end{itemize}