\documentclass{article}
\usepackage{graphicx} % Required for inserting images
\usepackage{ctex}

\usepackage{amsfonts}
\usepackage{amssymb}
\usepackage{amsmath}
\usepackage{amsthm}
% 导入 xparse 宏包以支持 LaTeX3 语法
\usepackage{xparse}
\usepackage{pgfplots}
% 用于插入带有坐标轴、标签和曲线的图像

% \begin{tikzpicture}
%     \begin{axis}[
%       xlabel=$x$,
%       ylabel=$f(x)$,
%       axis lines=middle,
%       xmin=-5, xmax=5,
%       ymin=-2, ymax=8,
%       width=\textwidth,
%       height=8cm
%     ]
%     \addplot[blue,domain=-3:3] {x^2};
%     \end{axis}
%   \end{tikzpicture}
\usepackage{tikz}
% 用于绘制一般的图像

% \begin{tikzpicture}[scale=0.8]
%     \draw[->] (-4,0) -- (4,0) node[right] {$x$};
%     \draw[->] (0,-1) -- (0,9) node[above] {$f(x)$};
%     \draw[domain=-3:3,smooth,variable=\x,blue] plot ({\x},{\x^2});
%   \end{tikzpicture}

\newtheorem{theorem}{Theorem}[subsection]
\newtheorem{lemma}{Lemma}[subsection]
\newtheorem{corollary}{corollary}[subsection]
\newtheorem{example}{Example}[subsection]
\newtheorem{definition}{Definition}[subsection]
% 为了证明中可以使用中文,后续定义证明时使用cproof而不是proof
\newenvironment{cproof}{%
{
    \textbf{Proof\\}
    }
}{
%   \hfill $\square$ 添加结束符号
%   \par\bigskip 可选的垂直间距
}
\newenvironment{exercise}[1]{%
{\textbf{Exercise #1} \\ 
    }
}{
  \hfill $\square$ 
  \par\bigskip 
}

\newenvironment{solution}{%
{
    \textbf{Solution\\}
    }
}{
  \hfill $\square$ 
  \par\bigskip 
}


% \newenvironment{identification}{%
% \heiti{定义}\kaishu
% }{%
% %   \hfill $\square$ 添加结束符号
% %   \par\bigskip 可选的垂直间距
% }

\newcommand{\RR}{\mathbb{R}}
\newcommand{\NN}{\mathbb{N}}
\newcommand{\CC}{\mathbb{C}}
\newcommand{\QQ}{\mathbb{Q}}
\newcommand{\ZZ}{\mathbb{Z}}
\newcommand{\FF}{\mathbb{F}}
\newcommand{\PP}{\mathbb{P}}
% 简化各种常见数的集合

\newcommand{\parameter}[1]{\left(#1\right)}

\newcommand{\bracket}[1]{\left[#1\right]}

% 各种自动变化大小的括号的简化

\newcommand{\ve}{\boldsymbol}
% 为了适应David C Lay线性代数中,简化斜体+粗体向量的书写

\newcommand{\base}{\mathcal}

\newcommand{\tb}{\textbf}

\newcommand{\col}{\text{Col}}

\newcommand{\row}{\text{Row}}
\newcommand{\nul}{\text{Nul}}
\newcommand{\spans}{\text{Span}}
\newcommand{\proj}{\text{proj}}
\newcommand{\adj}{\text{adj.}}
\newcommand{\rank}{\text{rank}}
\newcommand{\range}{\text{range}}
\newcommand{\n}{\text{null}}
\newcommand{\tr}{\text{tr}}
\newcommand{\sign}{\text{sign}}
\newcommand{\perm}{\text{perm}}
% 简化粗体字体的书写

\newcommand{\f}[2]{\frac}
\newcommand{\df}[2]{\dfrac}

\newcommand{\ip}[1]{\left<#1\right>}

\newcommand{\pa}{\paragraph}
\newcommand{\spa}{\subparagraph}
\newcommand{\se}{\section}
\newcommand{\sse}{\subsection}
\newcommand{\ssse}{\subsubsection}

% \NewDocumentCommand{\vs}{m m m}{
%     \ve{#1}_{#2},\cdots,\ve{#1}_{#3}
% }
% % 快速书写一个向量组,第一个参数为向量名称,后两个为首末角标

% % $\vs{b}{1}{n} $这是多参数命令的使用示例

% \NewDocumentCommand{\cvs}{m m m m m}{
%     #1_{#4}\ve{#2}_{#4} #3 \cdots #3 #1_{#5}\ve{#2}_{#5}
% }
% % 快速书写一个向量线性组合,第一个参数为系数,第二个参数为向量名称,第三个参数为运算符,后两个参数为角标

\NewDocumentCommand{\size}{m m}{
    #1\times#2
}


\title{习题二}
\author{徐海翁}
\date{2024.2.28}

\begin{document}

\begin{CJK}{UTF8}{gkai}

\maketitle
\tableofcontents

\begin{exercise}{6}
    不妨令$g(x) = \max\{f(x),0\},h(x) = \max\{-f(x),0\}$, $G(x) = \int_{a}^{x}g(x)\,dx , H(x) = \int_{a}^{x}h(x)\,dx$,我们可以验证
    \[f(x) = g(x) - h(x)\]
    和
    \[G'(x) = g(x) \geq 0, H'(x) = h(x) \geq 0\]
    从而有
    \[F(x) = \int_{a}^{x} f(x)\, dx = \int_{a}^{x} g(x)\, dx - \int_{a}^{x} h(x)\, dx = G(x) - H(x)\]
    因此我们证明了想要的结论
\end{exercise}

\begin{exercise}{7}
    将原等式移项得到
    \begin{equation}
        2 f(x) = 2 f(a) + \int_{a}^{x} f(x)\, dx
        \label{eq1}
    \end{equation}
    由于$f(x)\in C[a,b]$,故$\int_{a}^{x} f(x)\, dx$在$[a,b]$上可导,从而有$f(x)$在$[a,b]$上可导.

    对$(\ref{eq1})$左右两边求导得到
    \[2f'(x) = f(x)\]
    即有
    \[\parameter{\ln(f(x))}' = \dfrac{1}{2}\]
    从而
    \[\ln (f(x)) = \dfrac{1}{2}x + C\]
    即
    \[f(x) = e^{\frac{x}{2} + C}\]
    带入$x = a $时 $f(x) = f(a)$ 则得到
    \[f(x) = f(a) e^{\frac{x - a}{2}}\]
\end{exercise}

\begin{exercise}{8}
    \[\begin{aligned}
        \dfrac{d}{dx} \int_{-x^2}^{x^2} \sin^2 t\,dt &= \sin^2 (x^2) (x^2)' - \sin^2 (-x^2) (-x^2)' \\
        &= \sin^2 (x^2) 2x + \sin^2 (-x^2) 2x\\
        &= 4 x\sin^2 (x^2)\\
    \end{aligned}\]
\end{exercise}

\begin{exercise}{9}
    由条件有
    \[f(x) = f(x) - f(a) =\int_{a}^{x} f'(t) \, dt\]
    左右两式均取平方有
    \[f^2(x) = \parameter{\int_{a}^{x} f'(t) \, dt}^2\] 
    在两边从$a$到$b$积分,得
    \begin{equation}
        \int_{a}^{b}f^2(x)\, dx = \int_{a}^{b}\parameter{\int_{a}^{x} f'(t) \, dt}^2 \, dx
    \label{eq2}
    \end{equation}
    我们先考虑右式中内层积分,利用积分形式的柯西不等式
    \begin{equation}
        \begin{aligned}
            \int_{a}^{x} f'(t) \, dt &= \int_{a}^{x} (f'(t)\cdot 1) \, dt\\
            &\leq \parameter{\int_{a}^{x} [f'(t)]^2\, dt}^{\frac{1}{2}}\cdot \parameter{\int_{a}^{x} 1\, dt}^{\frac{1}{2}}\\    
            \label{eq3}
        \end{aligned}
    \end{equation}
    将(\ref{eq3})中结果代回(\ref{eq2})式,(\ref{eq2})式右侧可化为
    \begin{equation}
        \begin{aligned}
            \int_{a}^{b}\parameter{\int_{a}^{x} f'(t) \, dt}^2 \, dx &= \int_{a}^{b}\parameter{\parameter{\int_{a}^{x} [f'(t)]^2\, dt} \cdot\parameter{\int_{a}^{x} 1\, dt}}\, dx\\
            &= \int_{a}^{b}\parameter{\parameter{\int_{a}^{x} [f'(t)]^2\, dt} (x - a)}\, dx\\
            \label{eq4}
        \end{aligned}
    \end{equation}
    由于变上限积分$\parameter{\int_{a}^{x} [f'(t)]^2\, dt} \geq 0$且$b - a \geq x - a$
    \[\begin{aligned}
        \int_{a}^{b}\parameter{\parameter{\int_{a}^{x} [f'(t)]^2\, dt} (x - a)}\, dx &\leq (b - a) \int_{a}^{b}\parameter{\int_{a}^{x} [f'(t)]^2\, dt}\, dx\\
    \end{aligned}\]
    由于变上限积分$\parameter{\int_{a}^{x} [f'(t)]^2\, dt}$在$[a,b]$上单调递增,从而
    \[\int_{a}^{x} [f'(t)]^2\, dt \leq \int_{a}^{b} [f'(t)]^2\, dt\]
    上式中的右侧是一个与常量,我们将其代回(\ref{eq4})式,得到
    \begin{equation}
        \begin{aligned}
            (b - a) \int_{a}^{b}\parameter{\int_{a}^{x} [f'(t)]^2\, dt}\, dx &\leq (b - a) \int_{a}^{b}\parameter{\int_{a}^{b} [f'(t)]^2\, dt}\, dx\\
            &= (b - a)^2\int_{a}^{b} [f'(t)]^2\, dt\\
            &= (b - a)^2\int_{a}^{b} [f'(x)]^2\, dx\\
            \label{eq5}
            \end{aligned}
    \end{equation}
    整理上述式子(\ref{eq2})(\ref{eq3})(\ref{eq4})(\ref{eq5}),我们证明了
    \[\int_{a}^{b}f^2(x)\, dx \leq (b - a)^2\int_{a}^{b} [f'(x)]^2\, dx\]
\end{exercise}

\begin{exercise}{10}

    (1)令$g(x) = 1$,由第二积分中值定理可得,$\exists \theta \in [0,1]$
    \[
    \begin{aligned}
        \int_{0}^{1}f(x)\, dx &= \int_{0}^{1}f(x) g(x)\, dx \\
        &= f(0) \int_{0}^{\theta}g(x)\, dx + f(1) \int_{\theta}^{1} g(x)\, dx\\
        &= \theta f(0) + (1 - \theta)f(1)\\
    \end{aligned}   
    \]

    (2)由(1)中结论可知,存在$p \in [0,1]$使得
    \[\int_{0}^{1}f(x)\, dx = p f(0) + (1 - p)f(1)\]
    且不难利用$f(x)$严格单调下降证明$p \neq 0$.

    同样由于$f(x)$严格单调下降,$\int_{0}^{1}f(x)\, dx - f(1) > 0$\\

    我们考虑函数$\varphi(\theta) = 1 + \dfrac{\int_{0}^{1}f(x)\, dx - f(1)}{\theta}$,由前面的符号判断我们可知其在$(0,1]$上单调递减.\\

    这个函数在$(0,1]$是连续函数,并且利用$\varphi(p) = f(0) < c$[这是(1)中式子的变形],$\lim_{\theta\to 0+}\varphi(\theta) = +\infty > c$以及微分中值定理可知,$\exists \theta \in (0,p) \subset [0,1]$使得
    \[\int_{0}^{1}f(x)\, dx = \theta c + (1 - \theta)f(1)\]
\end{exercise}

\begin{exercise}{11}
    不妨记$I = \int_{0}^{a}u(x)\, dx + \int_{0}^{B}v(y)\, dy$,利用$v(u(x)) \equiv x$的性质可得
    \[\begin{aligned}
        I &= \int_{0}^{a}u(x)\, dx + \int_{0}^{v(B)} v(u(x)) d u(x)\\
        &= \int_{0}^{a}u(x)\, dx + \int_{0}^{v(B)} x d u(x)\\
        &= \int_{0}^{a}u(x)\, dx + (x u(x))|_{0}^{v(B)} - \int_{0}^{v(B)} u(x) dx\\
        &= \int_{0}^{a}u(x)\, dx + B v(B) - \int_{0}^{v(B)} u(x) dx\\
        &= B v(B) - \int_{a}^{v(B)} u(x)\, dx\\
    \end{aligned}\]
    由条件$0 \leq B \leq A$和单调性可知$v(B) \leq a$\\

    其中$a = v(B)$时,上式取等号;

    $a > v(B)$时, 由于$u(x)$单调性 $u(x) \geq 0$,故
    \[\int_{a}^{v(B)} u(x)\, dx \leq 0\]
    从而
    \[I = B v(B) - \int_{a}^{v(B)} u(x)\, dx \geq B v(B) \geq a B\]
    综上,待求证的式子成立
\end{exercise}

\end{CJK}
\end{document}

\begin{itemize} 
\end{itemize}