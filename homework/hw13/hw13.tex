\documentclass[11pt,hyperref,a4paper,UTF8]{ctexart}
\usepackage[left=2.50cm, right=2.50cm, top=2.50cm, bottom=2.50cm]{geometry}
\usepackage[unicode=true,colorlinks,urlcolor=blue,linkcolor=blue,bookmarksnumbered=true]{hyperref}
\usepackage{graphicx} % Required for inserting images
\usepackage{ctex}

\usepackage{amsfonts}
\usepackage{amssymb}
\usepackage{amsmath}
\usepackage{amsthm}
% 导入 xparse 宏包以支持 LaTeX3 语法
\usepackage{xparse}
\usepackage{pgfplots}
% 用于插入带有坐标轴、标签和曲线的图像

% \begin{tikzpicture}
%     \begin{axis}[
%       xlabel=$x$,
%       ylabel=$f(x)$,
%       axis lines=middle,
%       xmin=-5, xmax=5,
%       ymin=-2, ymax=8,
%       width=\textwidth,
%       height=8cm
%     ]
%     \addplot[blue,domain=-3:3] {x^2};
%     \end{axis}
%   \end{tikzpicture}
\usepackage{tikz}
% 用于绘制一般的图像

% \begin{tikzpicture}[scale=0.8]
%     \draw[->] (-4,0) -- (4,0) node[right] {$x$};
%     \draw[->] (0,-1) -- (0,9) node[above] {$f(x)$};
%     \draw[domain=-3:3,smooth,variable=\x,blue] plot ({\x},{\x^2});
%   \end{tikzpicture}

\newtheorem{theorem}{Theorem}[subsection]
\newtheorem{lemma}{Lemma}[subsection]
\newtheorem{corollary}{corollary}[subsection]
\newtheorem{example}{Example}[subsection]
\newtheorem{definition}{Definition}[subsection]
% 为了证明中可以使用中文,后续定义证明时使用cproof而不是proof
\newenvironment{cproof}{%
{
    \textbf{Proof\\}
    }
}{
%   \hfill $\square$ 添加结束符号
%   \par\bigskip 可选的垂直间距
}
\newenvironment{exercise}[1]{%
{\textbf{Exercise #1} \\ 
    }
}{
  \hfill $\square$ 
  \par\bigskip 
}

\newenvironment{solution}{%
{
    \textbf{Solution\\}
    }
}{
  \hfill $\square$ 
  \par\bigskip 
}


% \newenvironment{identification}{%
% \heiti{定义}\kaishu
% }{%
% %   \hfill $\square$ 添加结束符号
% %   \par\bigskip 可选的垂直间距
% }

\newcommand{\RR}{\mathbb{R}}
\newcommand{\NN}{\mathbb{N}}
\newcommand{\CC}{\mathbb{C}}
\newcommand{\QQ}{\mathbb{Q}}
\newcommand{\ZZ}{\mathbb{Z}}
\newcommand{\FF}{\mathbb{F}}
\newcommand{\PP}{\mathbb{P}}
% 简化各种常见数的集合

\newcommand{\parameter}[1]{\left(#1\right)}

\newcommand{\bracket}[1]{\left[#1\right]}

\newcommand{\abs}[1]{\left|#1\right|}
% 各种自动变化大小的括号的简化

\newcommand{\ve}{\boldsymbol}
% 为了适应David C Lay线性代数中,简化斜体+粗体向量的书写

\newcommand{\base}{\mathcal}

\newcommand{\tb}{\textbf}

\newcommand{\col}{\text{Col}}

\newcommand{\row}{\text{Row}}
\newcommand{\nul}{\text{Nul}}
\newcommand{\spans}{\text{Span}}
\newcommand{\proj}{\text{proj}}
\newcommand{\adj}{\text{adj.}}
\newcommand{\rank}{\text{rank}}
\newcommand{\range}{\text{range}}
\newcommand{\n}{\text{null}}
\newcommand{\tr}{\text{tr}}
\newcommand{\sign}{\text{sign}}
\newcommand{\perm}{\text{perm}}
% 简化粗体字体的书写

\newcommand{\f}[2]{\frac}
\newcommand{\df}[2]{\dfrac}

\newcommand{\ip}[1]{\left<#1\right>}

\newcommand{\pa}{\paragraph}
\newcommand{\spa}{\subparagraph}
\newcommand{\se}{\section}
\newcommand{\sse}{\subsection}
\newcommand{\ssse}{\subsubsection}

% \NewDocumentCommand{\vs}{m m m}{
%     \ve{#1}_{#2},\cdots,\ve{#1}_{#3}
% }
% % 快速书写一个向量组,第一个参数为向量名称,后两个为首末角标

% % $\vs{b}{1}{n} $这是多参数命令的使用示例

% \NewDocumentCommand{\cvs}{m m m m m}{
%     #1_{#4}\ve{#2}_{#4} #3 \cdots #3 #1_{#5}\ve{#2}_{#5}
% }
% % 快速书写一个向量线性组合,第一个参数为系数,第二个参数为向量名称,第三个参数为运算符,后两个参数为角标

\NewDocumentCommand{\size}{m m}{
    #1\times#2
}


\title{习题十三}
\author{徐海翁}
\date{2024.5.29}

\begin{document}

\begin{CJK}{UTF8}{gkai}

\maketitle
% \tableofcontents
\begin{exercise}{14}
    (1)转化为
    \[
        \begin{aligned}      
            \int_{\overset{\LARGE{\frown}}{AB}} y^2 dy &+ \int_{\overset{\LARGE{\frown}}{BC}} (2x + 6) dx + \int_{\overset{\LARGE{\frown}}{CD}} (9 + y^2) dy + \int_{\overset{\LARGE{\frown}}{DA}} 2xdx\\
            &= \int_{0}^2 y^2 dy + \int_{0}^3 (2x + 6) dx + \int_{2}^{0} (9 + y^2) dy + \int_{3}^0 2xdx\\
            &= \frac{8}{3} + 27 - \parameter{18 + \frac{8}{3}} - 9\\
            &= 0\\
        \end{aligned}
    \]
    (2)代入螺旋线的参数表示,原积分转化为
    \[\int_{0}^{3\pi} (4 + 2t^2)(-2\sin t) + (\sin (2\cos t) + e^{2\sin t}) dt\]

    (3)作代换
    \[\begin{cases}
        x = a\cos \theta\\
        y = b\sin \theta\\
    \end{cases}\]

    于是转化为
    \[
        \begin{aligned}
            \int_{-\frac{\pi}{2}}^{\frac{3}{2}\pi}\frac{a \cos \theta b \cos \theta - b \sin \theta (-a\sin\theta)}{a^2 \cos^2 \theta + b^2 \sin^2\theta}\, d\theta &= ab \int_{-\frac{\pi}{2}}^{\frac{3}{2}\pi} \frac{d\theta}{a^2 \cos^2 \theta + b^2 \sin^2\theta}\\
            &= 2ab \int_{-\frac{\pi}{2}}^{\frac{\pi}{2}} \frac{d\theta}{a^2 \cos^2 \theta + b^2 \sin^2\theta}\\
            &= 2ab \int_{-\frac{\pi}{2}}^{\frac{\pi}{2}} \frac{d(\tan \theta)}{a^2 + b^2 \tan^2\theta}\\
            &= 2ab \int_{-\infty}^{+\infty} \frac{dt}{a^2 + b^2 t^2}\\
            &= 2 \parameter{\arctan(1 + \frac{b}{a}t)}\bigg|_{-\infty}^{+\infty}\\
            &= 2\pi\\
        \end{aligned}
        \]

    (4)
    作代换
    \[
    \begin{cases}
        x = t\\
        y = t\\
        z = t\\
    \end{cases}\]
    原积分化为
    \[\int_{(0,0,0)}^{(1,1,1)} (1,2,3)\times (x,y,z) \cdot (dx,dy,dz) = \int_{0}^{1} t\bracket{(1,2,3)\times (1,1,1) \cdot (1,1,1)} dt = 0\]
\end{exercise}

\begin{exercise}{15}
    (1)考虑参数化为
    \[\begin{cases}
        x = z\cos \theta\\
        y = z\sin \theta\\
        z = z\\
    \end{cases}\]

    则我们可将被积函数视作
    \[P(x,y,z) = (y,0,z^2)\]
    \[\begin{cases}
        r_z = (\cos \theta,\sin \theta,1)\\
        r_\theta = (-z\sin \theta,z \cos \theta,0)\\
        r_z \times r_\theta = (-z\cos \theta,-z \sin \theta,z)\\
    \end{cases}\]

    \[
        \begin{aligned}    
            I &= \int_E P(x,y,z) \cdot d S\\
            &= \int_D (z \sin \theta,0,z^2) \cdot (-z\cos \theta,-z \sin \theta,z) dz d\theta\\
            &= \int_{0}^{2\pi} d\theta \int_{1}^{2} z^3 - z^2\sin\theta \cos \theta\\
            &= \frac{15}{2}\pi\\
        \end{aligned}
        \]

    (2)曲面为$x + y + z = 1$的限制,不妨取
    \[\begin{cases}
        x = u\\
        y = v\\
        z = 1 - u - v\\
    \end{cases},u \in [0,1],v \in [0, 1 - u]\]

    被积函数为
    \[P(x,y,z) = (x,0,0)\]
    \[
    \begin{aligned}
        I &= \int_S P \cdot n dS\\ 
        &= \int_D (u,0,0) \cdot \parameter{\frac{\sqrt{3}}{3},\frac{\sqrt{3}}{3},\frac{\sqrt{3}}{3}} d(u,v)\\
        &= \int_0^1 du \int_{0}^{1 - u} \frac{\sqrt{3}}{3}u dv\\
        &= \frac{\sqrt{3}}{18}\\
    \end{aligned}    
    \]
\end{exercise}

\begin{exercise}{16}
    利用对称性,我们只考虑第一象限的部分,我们考虑参数化
    \[\begin{cases}
        x = a^3 \cos^3 \theta\\
        y = a^3 \sin^3 \theta\\
    \end{cases}, \theta \in \bracket{0,\frac{\pi}{2}}
    \]

    不难有
    \[\begin{cases}
        dx = -3a^3 \cos^2 \theta \sin \theta d\theta\\
        dy = 3a^3 \sin^2 \theta \cos \theta d\theta\\
    \end{cases}\]

    于是就有
    \[S = 4 \int_{0}^{\frac{\pi}{2}} 3 a^6 (\cos^2 \theta + \sin^2 \theta) \sin^2 \theta \cos^2 \theta d\theta = 3a^6 \int_0^{\frac{\pi}{2}} \sin^2 2\theta d\theta = \frac{3a^6}{8}\pi\]
\end{exercise}

\end{CJK}
\end{document}

\begin{itemize} 
\end{itemize}