\documentclass{article}
\usepackage{graphicx} % Required for inserting images
\usepackage{ctex}

\usepackage{amsfonts}
\usepackage{amssymb}
\usepackage{amsmath}
\usepackage{amsthm}
% 导入 xparse 宏包以支持 LaTeX3 语法
\usepackage{xparse}
\usepackage{pgfplots}
% 用于插入带有坐标轴、标签和曲线的图像

% \begin{tikzpicture}
%     \begin{axis}[
%       xlabel=$x$,
%       ylabel=$f(x)$,
%       axis lines=middle,
%       xmin=-5, xmax=5,
%       ymin=-2, ymax=8,
%       width=\textwidth,
%       height=8cm
%     ]
%     \addplot[blue,domain=-3:3] {x^2};
%     \end{axis}
%   \end{tikzpicture}
\usepackage{tikz}
% 用于绘制一般的图像

% \begin{tikzpicture}[scale=0.8]
%     \draw[->] (-4,0) -- (4,0) node[right] {$x$};
%     \draw[->] (0,-1) -- (0,9) node[above] {$f(x)$};
%     \draw[domain=-3:3,smooth,variable=\x,blue] plot ({\x},{\x^2});
%   \end{tikzpicture}

\newtheorem{theorem}{Theorem}[subsection]
\newtheorem{lemma}{Lemma}[subsection]
\newtheorem{corollary}{corollary}[subsection]
\newtheorem{example}{Example}[subsection]
\newtheorem{definition}{Definition}[subsection]
% 为了证明中可以使用中文,后续定义证明时使用cproof而不是proof
\newenvironment{cproof}{%
{
    \textbf{Proof\\}
    }
}{
%   \hfill $\square$ 添加结束符号
%   \par\bigskip 可选的垂直间距
}
\newenvironment{exercise}[1]{%
{\textbf{Exercise #1} \\ 
    }
}{
  \hfill $\square$ 
  \par\bigskip 
}

\newenvironment{solution}{%
{
    \textbf{Solution\\}
    }
}{
  \hfill $\square$ 
  \par\bigskip 
}


% \newenvironment{identification}{%
% \heiti{定义}\kaishu
% }{%
% %   \hfill $\square$ 添加结束符号
% %   \par\bigskip 可选的垂直间距
% }

\newcommand{\RR}{\mathbb{R}}
\newcommand{\NN}{\mathbb{N}}
\newcommand{\CC}{\mathbb{C}}
\newcommand{\QQ}{\mathbb{Q}}
\newcommand{\ZZ}{\mathbb{Z}}
\newcommand{\FF}{\mathbb{F}}
\newcommand{\PP}{\mathbb{P}}
% 简化各种常见数的集合

\newcommand{\parameter}[1]{\left(#1\right)}

\newcommand{\bracket}[1]{\left[#1\right]}

\newcommand{\abs}[1]{\left|#1\right|}
% 各种自动变化大小的括号的简化

\newcommand{\ve}{\boldsymbol}
% 为了适应David C Lay线性代数中,简化斜体+粗体向量的书写

\newcommand{\base}{\mathcal}

\newcommand{\tb}{\textbf}

\newcommand{\col}{\text{Col}}

\newcommand{\row}{\text{Row}}
\newcommand{\nul}{\text{Nul}}
\newcommand{\spans}{\text{Span}}
\newcommand{\proj}{\text{proj}}
\newcommand{\adj}{\text{adj.}}
\newcommand{\rank}{\text{rank}}
\newcommand{\range}{\text{range}}
\newcommand{\n}{\text{null}}
\newcommand{\tr}{\text{tr}}
\newcommand{\sign}{\text{sign}}
\newcommand{\perm}{\text{perm}}
% 简化粗体字体的书写

\newcommand{\f}[2]{\frac}
\newcommand{\df}[2]{\dfrac}

\newcommand{\ip}[1]{\left<#1\right>}

\newcommand{\pa}{\paragraph}
\newcommand{\spa}{\subparagraph}
\newcommand{\se}{\section}
\newcommand{\sse}{\subsection}
\newcommand{\ssse}{\subsubsection}

% \NewDocumentCommand{\vs}{m m m}{
%     \ve{#1}_{#2},\cdots,\ve{#1}_{#3}
% }
% % 快速书写一个向量组,第一个参数为向量名称,后两个为首末角标

% % $\vs{b}{1}{n} $这是多参数命令的使用示例

% \NewDocumentCommand{\cvs}{m m m m m}{
%     #1_{#4}\ve{#2}_{#4} #3 \cdots #3 #1_{#5}\ve{#2}_{#5}
% }
% % 快速书写一个向量线性组合,第一个参数为系数,第二个参数为向量名称,第三个参数为运算符,后两个参数为角标

\NewDocumentCommand{\size}{m m}{
    #1\times#2
}


\title{习题三}
\author{徐海翁}
\date{2024.3.6}

\begin{document}

\begin{CJK}{UTF8}{gkai}

\maketitle
\tableofcontents

\begin{exercise}{12}    \\

    $p = 0$时,$I = 1,\tilde{I} = 1$\\

    $p = 1$时,$I = \frac{1}{2}$, $\tilde{I} = \frac{1}{2}$\\

    $p = 2$时,$I = \frac{1}{3}$, $\tilde{I} = \dfrac{65}{192}$\\

    $p = 3$时,$I = \frac{1}{4}$, $\tilde{I} = \dfrac{33}{128}$\\

    $p = 4$时,$I = \frac{1}{5}$, $\tilde{I} = \dfrac{31012}{147456} = \dfrac{7753}{36864}$\\

    在区间$[x_{i - 1},x_i]$中,我们不妨设$c = \dfrac{x_{i - 1} + x_i}{2},\Delta x_i = x_{i} - x_{i - 1}, h = \dfrac{\Delta x_i}{2}$,并假设
    \[\varPsi(h) = \int_{c - h}^{c + h}f(x)\, dx\]

    对以上函数连续求导,不难得到
    \[\varPsi'(h) = f(c + h) + f(c - h)\]
    \[\varPsi''(h) = f'(c + h) - f'(c - h)\]
    \[\varPsi''(0) = 0\]
    \[\varPsi^{(3)}(h) = f''(c + h) + f''(c - h) \leq 2 \max_{x \in [x_{i -1},x_i]} f''(x)\]
    我们有
    \[I_i = \int_{x_{i - 1}}^{x_i}f(x)\, dx = \varPsi(h)\]
    而我们近似的结果
    \[
    \begin{aligned}
        \tilde{I_i} &= \dfrac{x_{i} - x_{i - 1}}{2}\bracket{f\parameter{\dfrac{x_{i - 1} + x_{i}}{2} - \sqrt{\dfrac{2}{3}} \dfrac{x_i - x_{i - 1}}{2}} + f\parameter{\dfrac{x_{i - 1} + x_{i}}{2} + \sqrt{\dfrac{2}{3}} \dfrac{x_i - x_{i - 1}}{2}}}\\
        % &= h \bracket{f\parameter{c - \sqrt{\frac{2}{3}} h} + f\parameter{c + \sqrt{\frac{2}{3}} h}}\\
        &= h\cdot \varPsi'(\sqrt{\dfrac{2}{3}} h)\\
    \end{aligned}    
    \]

    而对$\varPsi(h)$在$h = 0$处展开可以得到
    \begin{equation}
        \varPsi(h) = \varPsi(0) + \varPsi'(0)h + \dfrac{\varPsi''(0)}{2} h^2 + \dfrac{1}{2}\int_{0}^{h}\varPsi^{(3)}(t)(h - t)^2 \, dt\\
        \label{eq1}
    \end{equation}
    而对$\varPsi'\parameter{\sqrt{\dfrac{2}{3}} h}$在$h = 0$处展开可以得到
    \begin{equation}
        \varPsi'\parameter{\sqrt{\dfrac{2}{3}} h} = \varPsi'(0) + \varPsi''(0) \sqrt{\dfrac{2}{3}} h + \int_{0}^{\sqrt{\frac{2}{3}} h}\varPsi^{(3)}(t)(\sqrt{\frac{2}{3}}h -t)\, dt\\
        \label{eq2}
    \end{equation}

    不难用(\ref{eq1}) $- h$ (\ref{eq2})得到
    \[
    \begin{aligned}
        \abs{I_i - \tilde{I_i}} &= \abs{\varPsi(h) - h\cdot \varPsi'\parameter{\sqrt{\frac{2}{3}} h} }\\
        &= \abs{\dfrac{1}{2}\int_{0}^{h}\varPsi^{(3)}(t)(h - t)^2 \, dt - h\int_{0}^{\sqrt{\frac{2}{3}} h}\varPsi^{(3)}(t)(\sqrt{\frac{2}{3}}h -t)\, dt} \\
        &= \abs{\dfrac{1}{2}\int_{0}^{h}\varPsi^{(3)}(t)(h - t)^2 \, dt - \sqrt{\frac{2}{3}}h\int_{0}^{h}\varPsi^{(3)}(\sqrt{\frac{2}{3}}t)(\sqrt{\frac{2}{3}}h - \sqrt{\frac{2}{3}}t)\, dt} \\  
        &=  \abs{\int_{0}^{h} \varPsi^{(3)}(t) (\frac{1}{2}h^2 - ht + \frac{1}{2}t^2) - \varPsi^{(3)}(\sqrt{\frac{2}{3}}t)(\frac{2}{3}h^2 - \frac{2}{3}ht)\,dt}\\
        &\leq \abs{\int_{0}^{h} \varPsi^{(3)}(t) (\frac{1}{2}h^2 - ht + \frac{1}{2}t^2)\,dt} + \abs{\int_{0}^{h}\varPsi^{(3)}(\sqrt{\frac{2}{3}}t)(\frac{2}{3}h^2 - \frac{2}{3}ht)\,dt}\\
        &\leq \int_{0}^{h} \max_{t\in[x_{i - 1},x_i]}\abs{\varPsi^{(3)}(t)} (\frac{7}{6}h^2 - \frac{5}{3}ht + \frac{1}{2}t^2)\, dt\\
        &= (\dfrac{7}{6}h^2 t - \dfrac{5}{6}h t^2+ \dfrac{1}{6} t^3)|_0^h \max_{t\in[x_{i - 1},x_i]}\abs{\varPsi^{(3)}(t)}\\
        &= \dfrac{1}{2}h^3 \max_{t\in[x_{i - 1},x_i]}\abs{\varPsi^{(3)}(t)}\\
        &= \frac{1}{8}\Delta x_i^3 \max_{x \in [a,b]} |f''(x)|\\
        % &\leq \dfrac{1}{2}\max_{t\in[x_{i - 1},x_i]}\abs{\varPsi^{(3)}(t)}\int_{0}^{h}(h - t)^2\,dt + \max_{t\in[x_{i - 1},x_i]}\abs{\varPsi^{(3)}(t)}\int_{0}^{\sqrt{\frac{2}{3}}h}(\sqrt{\frac{2}{3}}h - t)\,dt\\
        % &= (\frac{1}{6}h^3 + \frac{1}{3}h^2) 2\cdot \max_{x \in [a,b]} |f''(x)|\\
        % &= (\frac{1}{24}\Delta x_i^3 + \frac{1}{6}\Delta x_i^2) \max_{x \in [a,b]} |f''(x)|\\
    \end{aligned}
    \]
    对于整个区间$[a,b]$,如果采取等距分割可以得到
    % \[|I - \tilde{I}|\leq  (\frac{1}{24n^3} + \frac{1}{6n^2}) \max_{x \in [a,b]} |f''(x)| = O(\dfrac{1}{n^2})\]
    \[|I - \tilde{I}|\leq  \frac{1}{8n^2}\max_{x \in [a,b]} |f''(x)| = O(\dfrac{1}{n^2})\]
    因此可以认为这样的结果是二次收敛的
\end{exercise}

\begin{exercise}{1}
    (1)
    % 可知$\dfrac{1}{x^2 - 4}$的原函数为$F(x) = \dfrac{1}{4}\ln\dfrac{x - 2}{x + 2}$,利用牛顿-莱布尼茨公式
    \[\begin{aligned}
        \int_{0}^{+\infty}\dfrac{1}{x^2 - 4}\,dx &= \int_{0}^{2}\dfrac{1}{x^2 - 4} \,dx+ \int_{2}^{4}\dfrac{1}{x^2 - 4}\,dx + \int_{4}^{+\infty}\dfrac{1}{x^2 - 4}\,dx\\
        &= 2\int_{2}^{4}\dfrac{1}{x^2 - 4} + + \int_{4}^{+\infty}\dfrac{1}{x^2 - 4}\,dx\\
    \end{aligned}\]
    根据比较判别法,右侧第一项是瑕积分,发散,这是因为
    \[\lim_{x \to 2} \dfrac{x - 2}{x^2 - 4} = \dfrac{1}{4}\]
    而$\int_{2}^{4}\dfrac{1}{x - 2}$是一个发散的积分

    第二项收敛,故整个积分发散\\

    (2)
    \[\int_{0}^{+\infty}\dfrac{1}{x^4 - 4}\, dx =\dfrac{1}{4}\int_{0}^{+\infty}\dfrac{1}{x^2 - 2}\, dx - \dfrac{1}{4}\int_{0}^{+\infty}\dfrac{1}{x^2 + 2}\, dx\]

    其中第一项发散,讨论类似于(1)小问,第二项收敛,故整个积分发散\\

    (3)由于被积函数是奇函数
    \[V.P \int_{-\infty}^{+\infty} x \sin x^2 = \lim_{A \to +\infty} \int_{-A}^{A} f(x)\, dx = \lim_{A \to +\infty} \int_{0}^{A} f(x)\, dx + \int_{0}^{A} f(-x)\, dx = 0\]

    (4)被积函数的原函数是$F(x) = -\dfrac{1}{2}e^{-x^2} + C$,因此依据牛顿莱布尼茨公式可得
    \[\int_{0}^{ + \infty} xe^{-x^2} \, dx = F(+\infty) - F(0) = 0 - \dfrac{-1}{2} = \dfrac{1}{2}\]

    (5)注意到被积函数的原函数为$F(x) = \ln\abs{x + \sqrt{x^2 + 1}} + C$
    \[\int_{0}^{+\infty} \dfrac{1}{\sqrt{x^2 + 1}}\, dx = F(+\infty) - F(0) \]
    显然$\lim_{x\to +\infty}F(x)$这个极限不存在

    (6)
    \[\int_{2}^{+\infty} \sqrt{1 + x^2} - \sqrt{x^2 - 1}\, dx = \int_{2}^{+\infty} \dfrac{2}{\sqrt{x^2 + 1} + \sqrt{x^2 - 1}}\, dx\]
    这时利用比较判别法
    \[\lim_{x\to +\infty} \dfrac{2x}{\sqrt{x^2 + 1} + \sqrt{x^2 - 1}} = 1\]
    可知这个积分发散
\end{exercise}

\begin{exercise}{2}
    第一部分我们使用反证法:

    假设$x\to +\infty$时没有这样的极限,那么由极限条件的反面,$\exists \varepsilon_0 > 0,\forall X > 0,\exists x > X$,使得
    \[|f(x)| \geq \varepsilon_0\]
    由$f(x)$的一致连续性,对于$\varepsilon = \frac{1}{2}\varepsilon_0 > 0$,$\exists \delta >0$,使得$\forall x',x''  \subset [0,+\infty),|x' - x''| < \delta$ 有
    \[|f(x') - f(x'')| < \frac{1}{2}\varepsilon_0\]

    从而我们对$\varepsilon_1 = \dfrac{\varepsilon_0 \delta}{2},\forall M \geq 0$,令$X = M + 1$,并且按照前面的步骤取出$x_0$,不妨设$f(x_0) > 0$,那么有$\forall x \in U(x_0,\delta)$
    \[f(x) > f(x_0) - \frac{1}{2}\varepsilon_0 \geq \frac{1}{2}\varepsilon_0 > 0\]
    取$a,b$分别为$x_0 - \frac{1}{2}\delta,x_0 + \frac{1}{2}\delta$,但是
    \[\int_{x_0 - \frac{1}{2}\delta}^{x_0 + \frac{1}{2}\delta} f(x)\, dx > \frac{1}{2}\varepsilon_0\delta = \varepsilon_1\]
    这与广义积分的柯西收敛原理矛盾,也就是这个广义积分不收敛,与题目条件矛盾\\

    不能将一致连续换为连续,我们可以设计如下的函数$n \in \NN$
    \[f(x) = \begin{cases}
        2^n(x - n),x\in[n,n + \frac{1}{2^n})\\
        2 - 2^n(x - n),x\in[n + \frac{1}{2^n},n + \frac{2}{2^n})\\  
        0,otherwise\\
    \end{cases}\]
    不难发现这个函数不收敛,因为任意$M>0$,我们都能找到一个$x > M$,使得$f(x) = 1$.但是根据这个函数的几何图像,我们可以知道这个函数的广义积分收敛于$1$
\end{exercise}


\begin{exercise}{3}
    显然在定义域范围内有$1 + x^a \geq 1$,且被积函数恒大于零
    \[\int_{0}^{+\infty}\dfrac{dx}{(1 + x^2)(1 + x^a)}\leq \int_{0}^{+\infty}\dfrac{dx}{(1 + x^2)} = \arctan (+\infty) - \arctan 0 = \dfrac{\pi}{2}\]
    积分单调递增而有上界,从而一定收敛.接下来我们证明这个收敛值不随$a$的变化而变化\\

    进行变量代换$t = \dfrac{1}{x}$,则
    \[\int_{0}^{+\infty}\dfrac{dx}{(1 + x^2)(1 + x^a)} =-\int_{+\infty}^{0}\dfrac{1}{t^2}\dfrac{dt}{(1 + \frac{1}{t^2})(1 + \frac{1}{t^a})} = \int_{0}^{+\infty}\dfrac{dt}{(1 + t^2)(1 + t^{-a})}\]
    移项即有$\forall a \geq 1$
    \[\int_{0}^{+\infty}\dfrac{(x^a -1) dx}{(1 + x^2)(1 + x^a)} = 0\]
    对于两个不同的$a,b\in[1,+\infty)$,二者的积分式相减得到
    \[
    \begin{aligned}
        \int_{0}^{+\infty}\dfrac{dx}{(1 + x^2)(1 + x^a)} - \dfrac{dx}{(1 + x^2)(1 + x^b)} &= \int_{0}^{+\infty}\dfrac{(x^b - x^a)dx}{(1 + x^2)(1 + x^a)(1 + x^b)}\\
        &= \int_{0}^{+\infty}\dfrac{-1}{2}\dfrac{(x^a - 1)(1 + x^b) - (1 + x^a)(x^b - 1)}{(1 + x^2)(1 + x^a)(1 + x^b)}dx\\
        &=\int_{0}^{+\infty}\dfrac{(x^a -1) dx}{(1 + x^2)(1 + x^a)} - \dfrac{(x^b -1) dx}{(1 + x^2)(1 + x^b)}\\
        &=0\\
    \end{aligned}    
    \]
    从而我们证明了这个式子的值与$a$的取值无关
\end{exercise}


\begin{exercise}{4}
    (1)被积函数的原函数$F(x) = \dfrac{1}{4}\ln\dfrac{x - 2}{x + 2}$,
    \[\int_{0}^{2}\dfrac{1}{x^2 - 4}\, dx = \lim_{x\to 2
    ^-}F(x) - F(0)\]
    这个极限不存在,故原瑕积分发散\\

    (2)被积函数的原函数$F(x) = -\ln \cos x,x \in [0,\dfrac{\pi}{2})$
    \[\int_{0}^{\frac{\pi}{2}}\tan x \, dx = \lim_{x\to \frac{\pi}{2}} F(x) - F(0)\]
    这个极限不存在,原瑕积分发散\\

    (3)被积函数同样是一个奇函数
    \[
    \begin{aligned}
        V.P. \int_{-1}^{1} \dfrac{\sin x^2}{x}\, dx &= \lim_{\eta \to 0}\int_{-1}^{-\eta}\dfrac{\sin x^2}{x}\, dx +\int_{\eta}^{1} \dfrac{\sin x^2}{x}\, dx\\
        &= \lim_{\eta \to 0} -\int_{-\eta}^{-1}\dfrac{\sin x^2}{x}\, dx +\int_{\eta}^{1} \dfrac{\sin x^2}{x}\, dx\\
        &= \lim_{\eta \to 0} -\int_{\eta}^{1}\dfrac{\sin x^2}{-x}\, d(-x) +\int_{\eta}^{1} \dfrac{\sin x^2}{x}\, dx\\
        &= \lim_{\eta \to 0} -\int_{\eta}^{1}\dfrac{\sin x^2}{x}\, dx +\int_{\eta}^{1} \dfrac{\sin x^2}{x}\, dx\\
        &= 0\\
    \end{aligned}    
    \]

    (4)这题有问题,考虑把$x - 3$改成$3 - x$
    
    进行变量代换$x = 2 + \sin t$,从而
    \[\int_{1}^{2}\dfrac{1}{\sqrt{(x - 1)(3 - x)}}\, dx = \int_{-\frac{\pi}{2}}^{0}\dfrac{\cos t}{\cos t}\, dt = \dfrac{\pi}{2}\]

    (5)做变量代换$t = \ln x$,从而
    \[\int_{0}^{1}x^2 (\ln x)^2\, dx = \int_{-\infty}^{0} e^{3t} t^2 \, dt\]
    此时被积函数原函数$F(t) = \parameter{\dfrac{1}{3}t^2 - \dfrac{2}{9}t + \dfrac{2}{27}}e^{3t}$,从而有
    \[\int_{0}^{1}x^2 (\ln x)^2\, dx = F(0) - F(-\infty) = \dfrac{2}{27}\]
\end{exercise}


\end{CJK}
\end{document}

\begin{itemize} 
\end{itemize}