\documentclass[11pt,hyperref,a4paper,UTF8]{ctexart}
\usepackage[left=2.50cm, right=2.50cm, top=2.50cm, bottom=2.50cm]{geometry}
\usepackage[unicode=true,colorlinks,urlcolor=blue,linkcolor=blue,bookmarksnumbered=true]{hyperref}
\usepackage{graphicx} % Required for inserting images
\usepackage{ctex}

\usepackage{amsfonts}
\usepackage{amssymb}
\usepackage{amsmath}
\usepackage{amsthm}
% 导入 xparse 宏包以支持 LaTeX3 语法
\usepackage{xparse}
\usepackage{pgfplots}
% 用于插入带有坐标轴、标签和曲线的图像

% \begin{tikzpicture}
%     \begin{axis}[
%       xlabel=$x$,
%       ylabel=$f(x)$,
%       axis lines=middle,
%       xmin=-5, xmax=5,
%       ymin=-2, ymax=8,
%       width=\textwidth,
%       height=8cm
%     ]
%     \addplot[blue,domain=-3:3] {x^2};
%     \end{axis}
%   \end{tikzpicture}
\usepackage{tikz}
% 用于绘制一般的图像

% \begin{tikzpicture}[scale=0.8]
%     \draw[->] (-4,0) -- (4,0) node[right] {$x$};
%     \draw[->] (0,-1) -- (0,9) node[above] {$f(x)$};
%     \draw[domain=-3:3,smooth,variable=\x,blue] plot ({\x},{\x^2});
%   \end{tikzpicture}

\newtheorem{theorem}{Theorem}[subsection]
\newtheorem{lemma}{Lemma}[subsection]
\newtheorem{corollary}{corollary}[subsection]
\newtheorem{example}{Example}[subsection]
\newtheorem{definition}{Definition}[subsection]
% 为了证明中可以使用中文,后续定义证明时使用cproof而不是proof
\newenvironment{cproof}{%
{
    \textbf{Proof\\}
    }
}{
%   \hfill $\square$ 添加结束符号
%   \par\bigskip 可选的垂直间距
}
\newenvironment{exercise}[1]{%
{\textbf{Exercise #1} \\ 
    }
}{
  \hfill $\square$ 
  \par\bigskip 
}

\newenvironment{solution}{%
{
    \textbf{Solution\\}
    }
}{
  \hfill $\square$ 
  \par\bigskip 
}


% \newenvironment{identification}{%
% \heiti{定义}\kaishu
% }{%
% %   \hfill $\square$ 添加结束符号
% %   \par\bigskip 可选的垂直间距
% }

\newcommand{\RR}{\mathbb{R}}
\newcommand{\NN}{\mathbb{N}}
\newcommand{\CC}{\mathbb{C}}
\newcommand{\QQ}{\mathbb{Q}}
\newcommand{\ZZ}{\mathbb{Z}}
\newcommand{\FF}{\mathbb{F}}
\newcommand{\PP}{\mathbb{P}}
% 简化各种常见数的集合

\newcommand{\parameter}[1]{\left(#1\right)}

\newcommand{\bracket}[1]{\left[#1\right]}

\newcommand{\abs}[1]{\left|#1\right|}
% 各种自动变化大小的括号的简化

\newcommand{\ve}{\boldsymbol}
% 为了适应David C Lay线性代数中,简化斜体+粗体向量的书写

\newcommand{\base}{\mathcal}

\newcommand{\tb}{\textbf}

\newcommand{\col}{\text{Col}}

\newcommand{\row}{\text{Row}}
\newcommand{\nul}{\text{Nul}}
\newcommand{\spans}{\text{Span}}
\newcommand{\proj}{\text{proj}}
\newcommand{\adj}{\text{adj.}}
\newcommand{\rank}{\text{rank}}
\newcommand{\range}{\text{range}}
\newcommand{\n}{\text{null}}
\newcommand{\tr}{\text{tr}}
\newcommand{\sign}{\text{sign}}
\newcommand{\perm}{\text{perm}}
% 简化粗体字体的书写

\newcommand{\f}[2]{\frac}
\newcommand{\df}[2]{\dfrac}

\newcommand{\ip}[1]{\left<#1\right>}

\newcommand{\pa}{\paragraph}
\newcommand{\spa}{\subparagraph}
\newcommand{\se}{\section}
\newcommand{\sse}{\subsection}
\newcommand{\ssse}{\subsubsection}

% \NewDocumentCommand{\vs}{m m m}{
%     \ve{#1}_{#2},\cdots,\ve{#1}_{#3}
% }
% % 快速书写一个向量组,第一个参数为向量名称,后两个为首末角标

% % $\vs{b}{1}{n} $这是多参数命令的使用示例

% \NewDocumentCommand{\cvs}{m m m m m}{
%     #1_{#4}\ve{#2}_{#4} #3 \cdots #3 #1_{#5}\ve{#2}_{#5}
% }
% % 快速书写一个向量线性组合,第一个参数为系数,第二个参数为向量名称,第三个参数为运算符,后两个参数为角标

\NewDocumentCommand{\size}{m m}{
    #1\times#2
}


\title{习题九}
\author{徐海翁}
\date{2024.3.27}

\begin{document}

\begin{CJK}{UTF8}{gkai}

\maketitle
% \tableofcontents

\begin{exercise}{27}
    (1)我们构造拉格朗日乘子函数
    \[L = x^2 + y^2 - \lambda(xy - 1)\]

    求偏导
    \[\begin{cases}
        \frac{\partial L}{\partial x} = 2x - \lambda y = 0\\
        \frac{\partial L}{\partial y} = 2y - \lambda x = 0\\
        \frac{\partial L}{\partial \lambda} = xy - 1 = 0
    \end{cases}\]

    我们解这个这个方程组可得
    \[\begin{cases}
        x = 1\\
        y = 1\\
        \lambda = 2\\
    \end{cases} or
    \begin{cases}
        x = -1\\
        y = -1\\
        \lambda = 2\\
    \end{cases}\]

    我们接下来求二阶偏导,直接写出矩阵
    \[\begin{pmatrix}
        2&-\lambda\\
        -\lambda&2\\
    \end{pmatrix}\]

    其特征值为$2 - \lambda,2 + \lambda$.我们没办法直接判断其是否正定.

    我们对限制条件在临界点处进行微分,也就得到
    \[x dy + y dx = 0\]

    在这里就有
    \[dx + dy = 0\]

    进行参数表达
    \[\begin{cases}
        dx = dt\\
        dy = -dt\\
    \end{cases}\]

    进而可以将矩阵化为
    \[
    \begin{pmatrix}
        1&-1\\
    \end{pmatrix}
    \begin{pmatrix}
        2&-\lambda\\
        -\lambda&2\\
    \end{pmatrix}
    \begin{pmatrix}
        1\\
        -1\\
    \end{pmatrix}
    =
    4 + 2\lambda = 8\\
    \]
    这个矩阵是正定的,因此在$(-1,-1)$和$(1,1)$处取得极小值$2$.\\

    (2)
    我们考虑求$g = \ln f = a \ln x + b \ln y + c \ln z$的极值,构建拉格朗日乘子函数
    \[L = a \ln x + b \ln y + c \ln z - \lambda(x + y + z - 1)\]

    求一阶偏导数
    \[\begin{cases}
        \frac{\partial L}{\partial x} = \frac{a}{x} - \lambda = 0\\
        \frac{\partial L}{\partial y} = \frac{b}{y} - \lambda = 0\\
        \frac{\partial L}{\partial z} = \frac{c}{z} - \lambda = 0\\
        \frac{\partial L}{\partial \lambda} = x + y + z - 1 = 0\\
    \end{cases}\]

    解得
    \[\begin{cases}
        x = \frac{a}{a + b + c}\\
        y = \frac{b}{a + b + c}\\
        z = \frac{c}{a + b + c}\\
        \lambda = a + b + c\\
    \end{cases}\]

    我们求二阶导得到矩阵
    \[\begin{pmatrix}
        -\frac{a}{x^2}&0&0\\
        0&-\frac{b}{y^2}&0\\
        0&0&-\frac{c}{z^2}\\
    \end{pmatrix}\]

    这个矩阵已经是负定的了,因此我们直接可以判断$(\frac{a}{a + b + c},\frac{b}{a + b + c},\frac{c}{a + b + c})$是我们的极大值点,对应极大值
    \[f_{\max} = \parameter{\frac{a}{a + b + c}}^a \parameter{\frac{b}{a + b + c}}^b \parameter{\frac{c}{a + b + c}}^c\]

    (3)我们考虑$g = f^2 = 4x^2 + y^2 + z^2 + 4 + 4xy + 4xz - 8x + 2yz - 4y - 4z$的条件极值,构建乘子函数
    \[L = 4x^2 + y^2 + z^2 + 4 + 4xy + 4xz - 8x + 2yz - 4y - 4z - \lambda(y^2 - xz - 1) - \mu(z^2 - xy - 1)\]

    求一阶偏导数
    \[\frac{\partial L}{\partial x} = 8x + 4y + 4z - 8 + \lambda z + \mu y = 0\]
    \[\frac{\partial L}{\partial y} = 2y + 4x + 2z - 4 - 2\lambda y + \mu x = 0\]
    \[\frac{\partial L}{\partial z} = 2z + 4x + 2y - 4 + x\lambda - 2\mu z = 0\]
    \[\frac{\partial L}{\partial \lambda} = y^2 - xz - 1 = 0\]
    \[\frac{\partial L}{\partial \mu} = z^2 - xy - 1 = 0\]

    可以解得
    \[\begin{cases}
        x = 0\\
        y = 1\\
        z = 1\\
        \lambda = 0\\
        \mu = 0\\
    \end{cases}\]

    求二阶偏导数得到矩阵
    \[\begin{pmatrix}
        8&4 + \mu&4 + \lambda\\
        4 + \mu&2 - 2\lambda&2\\
        4 + \lambda&2&2 - 2\mu\\
    \end{pmatrix} = \begin{pmatrix}
        8&4&4\\
        4&2&2\\
        4&2&2\\
    \end{pmatrix}\]

    显然这个矩阵有$0$特征值,因此我们要考虑将变量进行参数表达.\\

    对两个约束关系式在临界点微分得到
    \[\begin{cases}
        2ydy - xdz - zdx = 0\\
        2zdz - xdy - ydx = 0\\
    \end{cases} \Rightarrow
    \begin{cases}
        2dy - dx = 0\\
        2dz - dx = 0\\
    \end{cases}
    \]

    不妨取
    \[\begin{cases}
        dx = 2dt\\
        dy = dt\\
        dz = dt\\
    \end{cases}\]

    从而有
    \[\begin{pmatrix}
        2&1&1\\
    \end{pmatrix}
    \begin{pmatrix}
        8&4&4\\
        4&2&2\\
        4&2&2\\
    \end{pmatrix}
    \begin{pmatrix}
        2\\
        1\\
        1\\
    \end{pmatrix}
    =
    72
    \]

    我们得到的矩阵是正定的,因此我们在临界点$(0,1,1)$处取到极小值$0$
\end{exercise}

\begin{exercise}{28}
    不妨设内接点中的一点为$(x,y,z)$,利用题目要求中各面均与坐标平面垂直以及该椭球的对称性可知其他对称点所在位置,这里为简便起见,利用对称性设$x,y,z > 0$,我们不难得到体积
    \[V = 2 x \cdot 2 y \cdot 2 z = 8 x y z\]

    由于内接点的约束(设$a,b,c > 0$)
    \[\frac{x^2}{a^2} + \frac{y^2}{4 a^2} + \frac{z^2}{b^2} = 1\]

    这个问题利用三元均值不等式可以立刻得出结果,但是这里我们使用条件极值的方式求解.\\

    首先构造拉格朗日乘子函数
    \[L(x,y,z,\lambda) = 8xyz - \lambda \parameter{\frac{x^2}{a^2} + \frac{y^2}{4 a^2} + \frac{z^2}{b^2} - 1}\]

    求一阶偏导数
    \[\frac{\partial L}{\partial x} = 8yz - \lambda \frac{2x}{a^2} = 0\]
    \[\frac{\partial L}{\partial y} = 8xz - \lambda \frac{y}{2a^2} = 0\]
    \[\frac{\partial L}{\partial z} = 8xy - \lambda \frac{2z}{b^2} = 0\]
    \[\frac{\partial L}{\partial \lambda} = \frac{x^2}{a^2} + \frac{y^2}{4 a^2} + \frac{z^2}{b^2} - 1 = 0\]

    解得
    \[\begin{cases}
        x = \frac{\sqrt{3}}{3} a\\

        y = 2\frac{\sqrt{3}}{3} a\\

        z = \frac{\sqrt{3}}{3} b\\

        \lambda = 8\frac{\sqrt{3}}{3} a^2 b\\
    \end{cases}\]

    之后我们继续考虑二阶偏导数,这里我们直接写出矩阵
    \[\begin{pmatrix}
        \frac{-2\lambda}{a^2}&8z&8y\\
        8z&\frac{-\lambda}{2a^2}&8x\\
        8y&8x&\frac{-2\lambda}{b^2}\\
    \end{pmatrix}\]

    我们在这个约束条件上作微分
    \[\frac{2xdx}{a^2} + \frac{ydy}{2 a^2} + \frac{2zdz}{b^2} = 0\]

    代入我们解出的临界点,可化简得到
    \[\frac{2}{a}dx + \frac{1}{a}dy + \frac{2}{b}dz = 0\]

    进行参数表达
    \[\begin{cases}
        dx = adt_1\\
        dy = 2adt_2\\
        dz = -bdt_1 - bdt_2 \\
    \end{cases}\]

    于是可以写出
    \[
    \begin{aligned} 
    \begin{pmatrix}
        a&0&-b\\
        0&2a&-b\\
    \end{pmatrix}
    \begin{pmatrix}
        \frac{-2\lambda}{a^2}&8z&8y\\
        8z&\frac{-\lambda}{2a^2}&8x\\
        8y&8x&\frac{-2\lambda}{b^2}\\
    \end{pmatrix}
    \begin{pmatrix}
        a&0\\
        0&2a\\
        -b&-b\\
    \end{pmatrix}
    &=
    \begin{pmatrix}
        -\frac{64\sqrt{3}}{3}a^2 b&-\frac{32\sqrt{3}}{3}a^2 b\\
        -\frac{32\sqrt{3}}{3}a^2 b&-\frac{64\sqrt{3}}{3}a^2 b\\
    \end{pmatrix}\\
    &=
    -\frac{32\sqrt{3}}{3}a^2 b
    \begin{pmatrix}
        2&1\\
        1&2\\
    \end{pmatrix}
    \end{aligned}   
    \]

    不难发现矩阵
    \[\begin{pmatrix}
        2&1\\
        1&2\\
    \end{pmatrix}\]

    的特征值是$1,3$,故整个矩阵的特征值为
    \[-\frac{32\sqrt{3}}{3}a^2 b,-32\sqrt{3}a^2 b\]

    这个矩阵是一个负定的矩阵,从而我们求出的$(x,y,z)$是极大值点,代入可得
    \[V = 8 \frac{\sqrt{3}}{3} a\cdot 2\frac{\sqrt{3}}{3} a\cdot \frac{\sqrt{3}}{3} b = \frac{16\sqrt{3}a^2 b}{9}\]
\end{exercise}

\begin{exercise}{29}
    我们设切点为$(x_0,y_0,z_0)$,在该点处求梯度可得
    \[\nabla F(x_0) = (\frac{2x_0}{a^2},\frac{2y_0}{b^2} ,- \frac{2z_0}{c^2})\]

    我们的切平面就满足
    \[(\ve{x} - \ve{x}_0)\cdot (\frac{2x_0}{a^2},\frac{2y_0}{b^2}, - \frac{2z_0}{c^2}) = 0\]

    \[\frac{x_0}{a^2}(x - x_0) + \frac{y_0}{b^2}(y - y_0) - \frac{z_0}{c^2}(z - z_0) = 0\]
    % 我们可选取曲面上的曲线$r_1(t) = (at,b,ct)$,其切向量为
    % \[r'_1(t) = (a,0,c)\]

    % 同理可取另一个曲线$r_2(t) = (a,bt,ct)$,其切向量为
    % \[r'_2(t) = (0,b,c)\]

    % 再取曲面上一个定点$(a,b,c)$,于是就有曲面
    % \[r = (a,b,c) + k(a,0,c) + h(0,b,c), k,h \in \RR\]
\end{exercise}

\begin{exercise}{30}
    题目这里的``曲线''应该改成``曲面''.\\

    在$(x_0,y_0,z_0)$处,我们固定$\varphi$不动,对$\theta$求偏导可得
    \[x_\theta = -a \sin \theta \sec \varphi\]
    \[y_\theta = b \cos \theta \sec \varphi\]
    \[z_\theta = 0\]

    固定$\theta$不动,求偏导可得
    \[x_\varphi = a \cos \theta \frac{\sin \varphi}{\cos^2 \varphi}\]
    \[y_\varphi = b \sin \theta \frac{\sin \varphi}{\cos^2 \varphi}\]
    \[z_\varphi = c \frac{1}{\cos^2 \varphi}\]

    于是我们在$(x_0,y_0,z_0)$处的切平面就可以表示为
    \[r(h,k) = (x_0,y_0,z_0) + h(-a \sin \theta \sec \varphi,b \cos \theta \sec \varphi,0) + k(a \cos \theta \frac{\sin \varphi}{\cos^2 \varphi},b \sin \theta \frac{\sin \varphi}{\cos^2 \varphi},c \frac{1}{\cos^2 \varphi})\]
\end{exercise}

\begin{exercise}{31}
    切线为
    \[
    \begin{aligned}    
        r'(\theta) &= \parameter{-a\sin \frac{\theta}{2} \parameter{1 + \frac{1}{2}\sec^2 \frac{\theta}{2}},b \cos \frac{\theta}{2} , \frac{c}{2}\sec^2 \frac{\theta}{2}}
    \end{aligned}    
    \]

    \[s(\theta) = \int_{0}^{\theta}\sqrt{x_\theta^2 + y_\theta^2 + z_\theta^2} \, d\theta = \frac{1}{2}\int_{0}^{\theta} \sqrt{b^2 \cos^2 \frac{\theta}{2} + 4 a^2\sin^2 \frac{\theta}{2} + 4a^2 \sin^2 \frac{\theta}{2} \sec^2 \frac{\theta}{2} + a^2 \sin^2 \sec^4 \frac{\theta}{2} + c^2 \sec^4 \frac{\theta}{2}}\, d\theta\]

    \[\ve{T} = \frac{dr/d\theta}{ds/d\theta} = \frac{r'(\theta)}{s'(\theta)} = \frac{\parameter{-a\sin \frac{\theta}{2} \parameter{2 + \sec^2 \frac{\theta}{2}},2b \cos \frac{\theta}{2} , c\sec^2 \frac{\theta}{2}}}{\sqrt{b^2 \cos^2 \frac{\theta}{2} + 4 a^2\sin^2 \frac{\theta}{2} + 4a^2 \sin^2 \frac{\theta}{2} \sec^2 \frac{\theta}{2} + a^2 \sin^2 \sec^4 \frac{\theta}{2} + c^2 \sec^4 \frac{\theta}{2}}}\]

    我们可以先求出$r''$
    \[r''(\theta) = \parameter{-\frac{a}{2}\cos \frac{\theta}{2} + \frac{a}{4}\sec \frac{\theta}{2} - \frac{a}{2}\sec^3 \frac{\theta}{2},-\frac{b}{2}\sin \frac{\theta}{2},\frac{c}{2}\sin \frac{\theta}{2} \sec^3 \frac{\theta}{2}}\]

    \[r' \times r'' = \parameter{\frac{3}{4}bc \sin \frac{\theta}{2} \sec^2 \frac{\theta}{2},\frac{3}{8}ac(-2\sec \frac{\theta}{2} + \sec^3 \frac{\theta}{2}),\frac{3}{4}ab\sec^2 \frac{\theta}{2}}\]
    % 同理利用我们课上的推导,可知

    \[\ve{N} = \frac{r''}{s'(\theta)} \cdot \frac{\|r'\|^3}{\|r'\times r''\|} = \frac{\|r'\| r''}{\|r'\times r''\|}\]

    代入可得(其中很多模长的计算代入范数并不能实际带来任何形式的简化,故保留模长的记号,同时$r''(\theta)$可参见上面未代入)
    \[\ve{N} = \frac{\sqrt{b^2 \cos^2 \frac{\theta}{2} + 4 a^2\sin^2 \frac{\theta}{2} + 4a^2 \sin^2 \frac{\theta}{2} \sec^2 \frac{\theta}{2} + a^2 \sin^2 \sec^4 \frac{\theta}{2} + c^2 \sec^4 \frac{\theta}{2}}}{\left\|\parameter{3bc \sin \frac{\theta}{2} \sec^2 \frac{\theta}{2},\frac{3}{2}ac(-2\sec \frac{\theta}{2} + \sec^3 \frac{\theta}{2}),
    3ab\sec^2 \frac{\theta}{2}}\right\|} r''(\theta)\]

    % \[\ve{B} = \ve{N}\times \ve{T} = \frac{\|r'\| r'' \times r'}{\|r'\times r''\|}\]
    \[\ve{B} = \frac{ r' \times r''}{\|r'\times r''\|}\]

    代入得到
    % \[
    %     \begin{aligned}
    %     \ve{B} = -\frac{1}{2} \cdot \frac{\sqrt{b^2 \cos^2 \frac{\theta}{2} + 4 a^2\sin^2 \frac{\theta}{2} + 4a^2 \sin^2 \frac{\theta}{2} \sec^2 \frac{\theta}{2} + a^2 \sin^2 \sec^4 \frac{\theta}{2} + c^2 \sec^4 \frac{\theta}{2}}}{\left\|\parameter{\frac{3}{4}bc \sin \frac{\theta}{2} \sec^2 \frac{\theta}{2},\frac{3}{8}ac(-2\sec \frac{\theta}{2} + \sec^3 \frac{\theta}{2}),\frac{3}{4}ab\sec^2 \frac{\theta}{2}}\right\|}(r'' \times r')
    % \end{aligned}
    % \]
    \[
        \begin{aligned}
        \ve{B} =  \frac{\parameter{\frac{3}{4}bc \sin \frac{\theta}{2} \sec^2 \frac{\theta}{2},\frac{3}{8}ac(-2\sec \frac{\theta}{2} + \sec^3 \frac{\theta}{2}),\frac{3}{4}ab\sec^2 \frac{\theta}{2}}}{\left\|\parameter{\frac{3}{4}bc \sin \frac{\theta}{2} \sec^2 \frac{\theta}{2},\frac{3}{8}ac(-2\sec \frac{\theta}{2} + \sec^3 \frac{\theta}{2}),\frac{3}{4}ab\sec^2 \frac{\theta}{2}}\right\|}
    \end{aligned}
    \]

\end{exercise}

\begin{exercise}{32}
    \[r'(t) = (-a\sin t,a \cos t,b)\]

    从而有
    \[s(t) = \int_{0}^{t}\sqrt{a^2 + b^2}\,dt = \sqrt{a^2 + b^2} t\]

    \[r(s) = (a \cos \frac{1}{\sqrt{a^2 + b^2}} s,a \sin \frac{1}{\sqrt{a^2 + b^2}} s,b \frac{1}{\sqrt{a^2 + b^2}} s)\]
    \[\dot{r} = \frac{1}{\sqrt{a^2 + b^2}}(-a \sin \frac{1}{\sqrt{a^2 + b^2}} s,a \cos \frac{1}{\sqrt{a^2 + b^2}} s,b) = \ve{T}\]
    \[\ddot{r} = \frac{1}{a^2 + b^2}(-a \cos \frac{1}{\sqrt{a^2 + b^2}} s,-a \sin \frac{1}{\sqrt{a^2 + b^2}} s,0)\]
    \[\ve{N} = \frac{\ddot{r}}{\|\ddot{r}\|} = (- \frac{a}{|a|} \cos \frac{1}{\sqrt{a^2 + b^2}} s,- \frac{a}{|a|} \sin \frac{1}{\sqrt{a^2 + b^2}} s,0)\]

    \[\ve{B} = \ve{N} \times \ve{T} = - \frac{a}{|a|} \frac{1}{\sqrt{a^2 + b^2}}(b \sin \frac{1}{\sqrt{a^2 + b^2}}s,-b \cos \frac{1}{\sqrt{a^2 + b^2}}s,a)\]

    \[\kappa = \|\ddot{r}\| = \frac{|a|}{a^2 + b^2}\]

    \[\dddot{r} = \frac{1}{(a^2 + b^2)^\frac{3}{2}}(a \sin \frac{1}{\sqrt{a^2 + b^2}} s,-a \cos \frac{1}{\sqrt{a^2 + b^2}} s,0)\]

    \[\tau = \frac{\dddot{r} \cdot (\dot{r} \times \ddot{r})}{\|\ddot{r}\|^2} = \frac{b}{a^2 + b^2}\]
\end{exercise}


\end{CJK}
\end{document}

\begin{itemize} 
\end{itemize}