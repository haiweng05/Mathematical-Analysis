\documentclass{article}
\usepackage{graphicx} % Required for inserting images
\usepackage{ctex}

\usepackage{amsfonts}
\usepackage{amssymb}
\usepackage{amsmath}
\usepackage{amsthm}
% 导入 xparse 宏包以支持 LaTeX3 语法
\usepackage{xparse}
\usepackage{pgfplots}
% 用于插入带有坐标轴、标签和曲线的图像

% \begin{tikzpicture}
%     \begin{axis}[
%       xlabel=$x$,
%       ylabel=$f(x)$,
%       axis lines=middle,
%       xmin=-5, xmax=5,
%       ymin=-2, ymax=8,
%       width=\textwidth,
%       height=8cm
%     ]
%     \addplot[blue,domain=-3:3] {x^2};
%     \end{axis}
%   \end{tikzpicture}
\usepackage{tikz}
% 用于绘制一般的图像

% \begin{tikzpicture}[scale=0.8]
%     \draw[->] (-4,0) -- (4,0) node[right] {$x$};
%     \draw[->] (0,-1) -- (0,9) node[above] {$f(x)$};
%     \draw[domain=-3:3,smooth,variable=\x,blue] plot ({\x},{\x^2});
%   \end{tikzpicture}

\newtheorem{theorem}{Theorem}[subsection]
\newtheorem{lemma}{Lemma}[subsection]
\newtheorem{corollary}{corollary}[subsection]
\newtheorem{example}{Example}[subsection]
\newtheorem{definition}{Definition}[subsection]
% 为了证明中可以使用中文,后续定义证明时使用cproof而不是proof
\newenvironment{cproof}{%
{
    \textbf{Proof\\}
    }
}{
%   \hfill $\square$ 添加结束符号
%   \par\bigskip 可选的垂直间距
}
\newenvironment{exercise}[1]{%
{\textbf{Exercise #1} \\ 
    }
}{
  \hfill $\square$ 
  \par\bigskip 
}

\newenvironment{solution}{%
{
    \textbf{Solution\\}
    }
}{
  \hfill $\square$ 
  \par\bigskip 
}


% \newenvironment{identification}{%
% \heiti{定义}\kaishu
% }{%
% %   \hfill $\square$ 添加结束符号
% %   \par\bigskip 可选的垂直间距
% }

\newcommand{\RR}{\mathbb{R}}
\newcommand{\NN}{\mathbb{N}}
\newcommand{\CC}{\mathbb{C}}
\newcommand{\QQ}{\mathbb{Q}}
\newcommand{\ZZ}{\mathbb{Z}}
\newcommand{\FF}{\mathbb{F}}
\newcommand{\PP}{\mathbb{P}}
% 简化各种常见数的集合

\newcommand{\parameter}[1]{\left(#1\right)}

\newcommand{\bracket}[1]{\left[#1\right]}

\newcommand{\abs}[1]{\left|#1\right|}
% 各种自动变化大小的括号的简化

\newcommand{\ve}{\boldsymbol}
% 为了适应David C Lay线性代数中,简化斜体+粗体向量的书写

\newcommand{\base}{\mathcal}

\newcommand{\tb}{\textbf}

\newcommand{\col}{\text{Col}}

\newcommand{\row}{\text{Row}}
\newcommand{\nul}{\text{Nul}}
\newcommand{\spans}{\text{Span}}
\newcommand{\proj}{\text{proj}}
\newcommand{\adj}{\text{adj.}}
\newcommand{\rank}{\text{rank}}
\newcommand{\range}{\text{range}}
\newcommand{\n}{\text{null}}
\newcommand{\tr}{\text{tr}}
\newcommand{\sign}{\text{sign}}
\newcommand{\perm}{\text{perm}}
% 简化粗体字体的书写

\newcommand{\f}[2]{\frac}
\newcommand{\df}[2]{\dfrac}

\newcommand{\ip}[1]{\left<#1\right>}

\newcommand{\pa}{\paragraph}
\newcommand{\spa}{\subparagraph}
\newcommand{\se}{\section}
\newcommand{\sse}{\subsection}
\newcommand{\ssse}{\subsubsection}

% \NewDocumentCommand{\vs}{m m m}{
%     \ve{#1}_{#2},\cdots,\ve{#1}_{#3}
% }
% % 快速书写一个向量组,第一个参数为向量名称,后两个为首末角标

% % $\vs{b}{1}{n} $这是多参数命令的使用示例

% \NewDocumentCommand{\cvs}{m m m m m}{
%     #1_{#4}\ve{#2}_{#4} #3 \cdots #3 #1_{#5}\ve{#2}_{#5}
% }
% % 快速书写一个向量线性组合,第一个参数为系数,第二个参数为向量名称,第三个参数为运算符,后两个参数为角标

\NewDocumentCommand{\size}{m m}{
    #1\times#2
}


\title{习题五}
\author{徐海翁}
\date{2024.3.20}

\begin{document}

\begin{CJK}{UTF8}{gkai}

\maketitle
\tableofcontents

\begin{exercise}{9.3}
    利用柯西列的性质,也就是
    $\forall \varepsilon > 0$,$\exists N > 0$,$\forall n > N,p$使得
    \[\|x_n - x_{n + p}\| < \varepsilon\]

    利用距离的三角不等式
    $|x_n^1 - x_{n + p}^1| \leq |\|x_n - x_{n + p}\| < \varepsilon$.
    而由于我们知道对于实数空间$\RR$,柯西列都是收敛列,也就是序列$\{x_n^1\}$是收敛列,设其收敛于$x^1$,同理设$\{x_n^2\}$收敛于$x^2$,这样也就有

    $\forall \varepsilon > 0, \exists N > 0,\forall n > N$,使得
    \[|x_n^1 - x^1| < \sqrt{\frac{1}{2}} \varepsilon\]
    同理
    $\forall \varepsilon > 0, \exists N > 0,\forall n > N$,使得
    \[|x_n^2 - x^2| < \sqrt{\frac{1}{2}} \varepsilon\]
    这样利用勾股定理就有
    $\forall \varepsilon > 0, \exists N > 0,\forall n > N$,使得
    \[\|x_n - (x^1,x^2)\| < \varepsilon\]
    即这个点列收敛
\end{exercise}

\begin{exercise}{9.4}
    设这个聚点为$a$,根据聚点的定义,$\forall \varepsilon > 0$,$\exists x \in U_0(a,\varepsilon)\cap A$,故我们可以构造去心邻域$U_0(a,\frac{1}{n})(n = 1,2,\ldots)$,并在每个邻域与$A$的交集中取出$x_n$,使得
    \[\|x_n - a\| < \frac{1}{n}\]
    当$n \to +\infty$时,我们有$x_n \to a$,也就是$A$中存在点列$\{x_n\}$收敛到$a$.
\end{exercise}

\begin{exercise}{9.5}
    这里我们假定闭集的定义方式是``$F \subset X$,任意的收敛点列$\{x_n\}$的极限都在集合中'',否则证明命题的后半部分即是定义\\

    我们先证明命题的后半部分:

    对于某个开集$G$,记$F = X \setminus G$,我们考察$F$中的收敛点列$\{x_n\}$,假如有~$\lim_{n \to +\infty} x_n = a \notin F$,则有
    \[a \in X \setminus F = G\]
    由于开集的性质,$\exists \varepsilon >0$使得
    \[U(a,\varepsilon) \subset X \setminus F\]
    与此同时,当$n$充分大时,有
    \[x_n \in U(a,\varepsilon) \subset X \setminus F\]
    这与我们的点列$\{x_n\} \in F$矛盾,于是我们证明略开集的补集是一个闭集

    对于命题的前半部分,对于某个闭集$F$,我们记$G = X \setminus F$,对于$G$中任何一点$b$,取邻域$U(b,\frac{1}{n})$,如果对于任意$n > 0$,我们都有$U(b,\frac{1}{n}) \cap F \neq \varnothing$,那么我们能够在$F$中选出一个收敛于$b$的点列,但这与$F$是闭集的事实矛盾,从而一定存在一个$N > 0$,使得$\forall n > N$都有$U(b,\frac{1}{n}) \subset G$,由于$b$的任意性,我们证明了闭集的补集是一个开集
\end{exercise}

\begin{exercise}{9.6}
    (1)做变量代换$r = \sqrt{x^2 + y^2}$,当$(x,y)\to (0,0)$时必然有$r \to 0$,这样
    \[\lim_{(x,y)\to(0,0)} \frac{x^2 + y^2}{\sqrt{x^2 + y^2} - 2} = \lim_{r \to 0}\frac{r^2}{r - 2} = 0\]

    (2)做变量代换$r = \pi - z$,则
    \[
    \begin{aligned}
        \lim_{(x,y,z)\to(0,0,\pi)} \abs{\frac{\sin^3 z}{x^2 + y^2 + (\pi - z)^2}} &= \lim_{(x,y,r)\to(0,0,0)} \abs{\frac{\sin^3 r}{x^2 + y^2 + r^2}}\\
        &\leq \lim_{(x,y,r)\to(0,0,0)} \abs{\frac{r^3}{x^2 + y^2 + r^2}} \leq 0
    \end{aligned}    
    \]
    从而
    \[\lim_{(x,y,z)\to(0,0,\pi)} \frac{\sin^3 z}{x^2 + y^2 + (\pi - z)^2} = 0\]

    (3)不难利用复合函数的极限和函数连续性证明
    \[\lim_{(x,y)\to(1,1)} \ln(x + 2y) = \ln 3\]
    和
    \[\lim_{(x,y)\to(1,1)} \frac{1}{x + y} = \frac{1}{2}\]
    再不难利用两个函数乘积极限的性质有
    \[\lim_{(x,y)\to(1,1)} \frac{\ln(x + 2y)}{x + y} = \lim_{(x,y)\to(1,1)} \ln(x + 2y)  \cdot \lim_{(x,y)\to(1,1)} \frac{1}{x + y} = \frac{\ln 3}{2}\]
    
    (4)不妨设$x > 1,y > 1$,取$r = x + y$,那么$(x,y)\to(+\infty,+\infty)$,时有$r \to +\infty$此时利用一元函数的性质,我们有
    \[\lim_{r \to +\infty}\frac{r^4}{e^r} = 0\]
    同时
    \[
    \begin{aligned}
    \lim_{(x,y)\to(+\infty,+\infty)} \frac{x^2 + y^4}{e^{x+ y}} &\leq \lim_{(x,y)\to(+\infty,+\infty)} \frac{x^4 + 4x^3 y + 6 x^2 y^2 + 4 xy^3+ y^4}{e^{x+ y}}\\
    &= \lim_{r \to +\infty}\frac{r^4}{e^r} = 0     \\   
    \end{aligned}    
    \]
    又由于被求极限式子不小于$0$,从而原极限
    \[\lim_{(x,y)\to(+\infty,+\infty)} \frac{x^2 + y^4}{e^{x+ y}} = 0\]
\end{exercise}

\begin{exercise}{9.7}
    (1)重极限不存在,因为当我们令点以曲线$y = x^{\frac{3}{2}}$趋近于$(0,0)$时,$f(x,y) = \frac{1}{2}$,而令点以$y = x$趋近于$(0,0)$时,我们有$f(x,y) = \frac{x}{x^2 + 1}$,对于上述两种曲线上收敛于$(0,0)$的点列,它们具有两种不同的极限$\frac{1}{2}$和$0$,故原重积分不存在\\

    \[\lim_{x \to 0} \lim_{y \to 0} \frac{x^3 y^2}{x^6 + y^4} = \lim_{x \to 0}  0 = 0\]
    \[\lim_{y \to 0} \lim_{x \to 0} \frac{x^3 y^2}{x^6 + y^4} = \lim_{y \to 0}  0 = 0\]

    (2)
    \[\lim_{(x,y)\to(0,0)}\abs{(x+y)\sin\frac{1}{x}\sin\frac{1}{y}} \leq \lim_{(x,y)\to(0,0)}\abs{(x+y)} = 0\]
    故原重极限存在,值为$0$

    \[\lim_{x \to 0} \lim_{y \to 0} (x + y)\sin\frac{1}{x}\sin \frac{1}{y} = \lim_{x \to 0} \sin\frac{1}{x} \lim_{y \to 0} (x + y)\sin \frac{1}{y}\]
    由于极限
    \[\lim_{y \to 0} (x + y)\sin \frac{1}{y}\]
    不存在,故整个累次极限也不存在

    同理根据对称性,累次极限
    \[\lim_{y \to 0} \lim_{x \to 0} (x + y)\sin\frac{1}{x}\sin \frac{1}{y} \]
    也不存在
\end{exercise}

\begin{exercise}{9.8}
    对于一元函数$g(r) = \sin r$,其连续性我们我们已经在一元函数部分进行证明,在这里省略

    由于$p(x) = x^2$的连续性,$\forall \varepsilon = \varepsilon_0 > 0$,$\exists \delta_1 > 0$,使得$\forall x \in U(x_0,\delta_1)$时
    \[|p(x) - p(x_0)| < \frac{1}{2}\varepsilon\]
    由于$q(x) = x$的连续性,$\forall \varepsilon = \varepsilon_0 > 0$,$\exists \delta_2 > 0$,使得$\forall x \in U(x_0,\delta_2)$时
    \[|q(x) - q(x_0)| < \frac{1}{2}\varepsilon\]
    对于$r = h(x,y) = x^2 + y$,任取$a = (x_0,y_0)$,$\forall \varepsilon = \varepsilon_0 > 0$,$\exists \delta = \min\{\delta_1,\delta_2\}$,不妨限定$\delta < 1$
    当$(x,y) \in U(a,\delta)$时
    \[|h(x,y) - h(x_0,y_0)| = |x^2 - x_0^2 + y - y_0| \leq |x^2 - x_0^2| + |y - y_0| < \varepsilon\]
    因此我们也证明了$h(x,y)$的连续性

    由于$f = g \circ h$,以及后面两个函数的连续性,我们有$\forall \varepsilon > 0$,$\exists \eta > 0$使得$\forall r, |r  - r_0| < \eta$,有
    \[|g(r) - g(r_0)| < \varepsilon\]
    我们有对$\eta$,$\exists \delta >0$,使得$(x,y) \in U(a,\delta)$时有
    \[|h(x,y) - h(x_0,y_0)| < \eta\]
    故$(x,y) \in U(a,\delta)$时有
    \[|f(x,y) - f(x_0,y_0)| < \varepsilon\]
    由于我们前面所取$a = (x_0,y_0)$的任意性,我们证明了上述函数$f$时在全空间上的连续函数
\end{exercise}

\begin{exercise}{9}
    根据利普西茨连续的性质$\forall \varepsilon > 0$,$\exists \delta = \frac{\varepsilon}{2L}$,使得对于$\forall (x,y) \in U((x_0,y_0),\delta)$
    都有
    \[|f(x,y) - f(x_0,y_0)| \leq L (|x - x_0| + |y - y_0|)| \leq L (\frac{\varepsilon}{2L} + \frac{\varepsilon}{2L}) = \varepsilon\]
    其中第二个不等号利用了
    \[|x - x_0| \leq \|(x,y) - (x_0,y_0)\| ,|y - y_0| \leq \|(x,y) - (x_0,y_0)\|\]
\end{exercise}

\begin{exercise}{10}
    任取曲线$D: y = \alpha x^3$上的点$(x_0,y_0) = (t_0,\alpha t_0^3) \neq (0,0)$
    \[\lim_{{D \owns (x,y)\to (x_0,y_0)}} f(x,y) = \lim_{{D \owns t\to t_0}} \frac{\alpha t^6}{(1 + \alpha^2)t^6} = \frac{\alpha t_0^6}{(1 + \alpha^2)t_0^6} = f(x_0,y_0)\]
    其中中间的等号利用了一元函数的连续性

    而对于$(x_0,y_0) = (t_0,\alpha t_0^3) = (0,0)$
    \[\lim_{{D \owns (x,y)\to (x_0,y_0)}} f(x,y) = \lim_{{D \owns t\to t_0} } \frac{\alpha t^6}{(1 + \alpha^2)t^6} = \frac{\alpha}{(1 + \alpha^2)} \neq 0 = f(x_0,y_0)\]    

    从而$f(x,y)$在曲线$D$上连续,但在$(0,0)$处不连续
    
\end{exercise}

\end{CJK}
\end{document}

\begin{itemize} 
\end{itemize}