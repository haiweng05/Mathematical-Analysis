\documentclass{article}
\usepackage{graphicx} % Required for inserting images
\usepackage{ctex}

\usepackage{amsfonts}
\usepackage{amssymb}
\usepackage{amsmath}
\usepackage{amsthm}
% 导入 xparse 宏包以支持 LaTeX3 语法
\usepackage{xparse}
\usepackage{pgfplots}
% 用于插入带有坐标轴、标签和曲线的图像

% \begin{tikzpicture}
%     \begin{axis}[
%       xlabel=$x$,
%       ylabel=$f(x)$,
%       axis lines=middle,
%       xmin=-5, xmax=5,
%       ymin=-2, ymax=8,
%       width=\textwidth,
%       height=8cm
%     ]
%     \addplot[blue,domain=-3:3] {x^2};
%     \end{axis}
%   \end{tikzpicture}
\usepackage{tikz}
% 用于绘制一般的图像

% \begin{tikzpicture}[scale=0.8]
%     \draw[->] (-4,0) -- (4,0) node[right] {$x$};
%     \draw[->] (0,-1) -- (0,9) node[above] {$f(x)$};
%     \draw[domain=-3:3,smooth,variable=\x,blue] plot ({\x},{\x^2});
%   \end{tikzpicture}

\newtheorem{theorem}{Theorem}[subsection]
\newtheorem{lemma}{Lemma}[subsection]
\newtheorem{corollary}{corollary}[subsection]
\newtheorem{example}{Example}[subsection]
\newtheorem{definition}{Definition}[subsection]
% 为了证明中可以使用中文,后续定义证明时使用cproof而不是proof
\newenvironment{cproof}{%
{
    \textbf{Proof\\}
    }
}{
%   \hfill $\square$ 添加结束符号
%   \par\bigskip 可选的垂直间距
}
\newenvironment{exercise}[1]{%
{\textbf{Exercise #1} \\ 
    }
}{
  \hfill $\square$ 
  \par\bigskip 
}

\newenvironment{solution}{%
{
    \textbf{Solution\\}
    }
}{
  \hfill $\square$ 
  \par\bigskip 
}


% \newenvironment{identification}{%
% \heiti{定义}\kaishu
% }{%
% %   \hfill $\square$ 添加结束符号
% %   \par\bigskip 可选的垂直间距
% }

\newcommand{\RR}{\mathbb{R}}
\newcommand{\NN}{\mathbb{N}}
\newcommand{\CC}{\mathbb{C}}
\newcommand{\QQ}{\mathbb{Q}}
\newcommand{\ZZ}{\mathbb{Z}}
\newcommand{\FF}{\mathbb{F}}
\newcommand{\PP}{\mathbb{P}}
% 简化各种常见数的集合

\newcommand{\parameter}[1]{\left(#1\right)}

\newcommand{\bracket}[1]{\left[#1\right]}

\newcommand{\abs}[1]{\left|#1\right|}
% 各种自动变化大小的括号的简化

\newcommand{\ve}{\boldsymbol}
% 为了适应David C Lay线性代数中,简化斜体+粗体向量的书写

\newcommand{\base}{\mathcal}

\newcommand{\tb}{\textbf}

\newcommand{\col}{\text{Col}}

\newcommand{\row}{\text{Row}}
\newcommand{\nul}{\text{Nul}}
\newcommand{\spans}{\text{Span}}
\newcommand{\proj}{\text{proj}}
\newcommand{\adj}{\text{adj.}}
\newcommand{\rank}{\text{rank}}
\newcommand{\range}{\text{range}}
\newcommand{\n}{\text{null}}
\newcommand{\tr}{\text{tr}}
\newcommand{\sign}{\text{sign}}
\newcommand{\perm}{\text{perm}}
% 简化粗体字体的书写

\newcommand{\f}[2]{\frac}
\newcommand{\df}[2]{\dfrac}

\newcommand{\ip}[1]{\left<#1\right>}

\newcommand{\pa}{\paragraph}
\newcommand{\spa}{\subparagraph}
\newcommand{\se}{\section}
\newcommand{\sse}{\subsection}
\newcommand{\ssse}{\subsubsection}

% \NewDocumentCommand{\vs}{m m m}{
%     \ve{#1}_{#2},\cdots,\ve{#1}_{#3}
% }
% % 快速书写一个向量组,第一个参数为向量名称,后两个为首末角标

% % $\vs{b}{1}{n} $这是多参数命令的使用示例

% \NewDocumentCommand{\cvs}{m m m m m}{
%     #1_{#4}\ve{#2}_{#4} #3 \cdots #3 #1_{#5}\ve{#2}_{#5}
% }
% % 快速书写一个向量线性组合,第一个参数为系数,第二个参数为向量名称,第三个参数为运算符,后两个参数为角标

\NewDocumentCommand{\size}{m m}{
    #1\times#2
}


\title{习题六}
\author{徐海翁}
\date{2024.3.27}

\begin{document}

\begin{CJK}{UTF8}{gkai}

\maketitle
\tableofcontents


\begin{exercise}{11}
    $\Rightarrow$,对于$[a,b]$上的任意一个收敛序列,设其极限为$x_0$,由于$f(x)\in C[a,b]$,故$\forall x_0 \in [a,b]$,对$\forall \{x_n\}$满足$\lim_{n \to \infty} x_n = x_0$,都有$\lim_{n\to \infty} f(x_n) = f(x_0)$,也就是
    \[\lim_{n\to \infty} (x_n,f(x_n)) = (x_0,f(x_0))\]
    又由于$\RR^2$上的任何一个收敛点列$\{(x_n,f(x_n))\}$都有其第一个分量组成的点列$\{x_n\}$是$[a,b]$上的收敛点列,上面的过程说明了这个$\RR^2$上的序列收敛于$(x_0,f(x_0)) \in G$,其中$x_0 \in [a,b]$是$\{x_n\}$的极限.
    
    由于前面收敛序列的任意性,以及闭集等价定义,可知$G$是闭集\\

    $\Leftarrow$,若$G$是一个闭集,则其上任意一个收敛序列$\{(x_n,f(x_n))\}$的极限都在$G$内,
    我们考虑$[a,b]$上收敛到$x_0 \in [a,b]$的任何一个收敛序列$\{x_n\}$,根据$G$的定义,我们有$(x_0,f(x_0)) \in G$,又由于$x_n \in [a,b]$,而$G$是闭集,$(x_n,f(x_n))$收敛到
    \[(x_0,\lim_{n\to \infty}f(x_n)) \in G\]
    
    利用函数的单值性,可知
    \[\lim_{n\to \infty}f(x_n) = f(x_0)\]

    由于我们前面取了收敛于$x_0$任意一点的任意点列,且这样的$x_0$是在$[a,b]$上任取的,故在$[a,b]$上任意一点我们都有$f$连续,即$f(x)\in C[a,b]$
\end{exercise}

\begin{exercise}{1}
    (1)
    \[f'_x(x,y) = \sin(x^2 + y^2) + 2x^2 \cos(x^2 + y^2)\]
    \[f'_y(x,y) = 2xy\cos(x^2 + y^2)\]

    (2)
    \[f'_x(x,y) = \frac{2x}{(1 + x^2)y},f'_y(x,y) = -\frac{\ln(1 + x^2)}{y^2}\]

    (3)
    \[f'_x(x,y) = e^x \sin(ky),f'_y(x,y) = k e^x \cos (ky)\]

    (4)
    \[f'_x(x,y,z) = 2x \ln(|z(1 + y^2)|)\]
    \[f'_y(x,y,z) = x^2 \frac{2y}{1 + y^2}\]
    \[f'_z(x,y,z) = x^2 \frac{1}{z}\]

    (5)
    \[f'_x(x,y) = \frac{1}{1 + (\frac{1 + xy}{x^2 + y^2})^2} \frac{y(x^2 + y^2) - (1 + xy) 2x}{(x^2 + y^2)} = \frac{-x^2y + y^3 - 2x}{(x^2 + y^2)^2 + (1 + xy)^2}\]
    利用对称性可以立刻得到
    \[f'_y(x,y) = \frac{-xy^2 + x^3 - 2y}{(x^2 + y^2)^2 + (1 + xy)^2}\]
\end{exercise}

\begin{exercise}{2}
    (1)我们先求偏导数:
    \[f'_x(x,y) = -y^2 \sin(xy^2) - \frac{2x}{x^2 + y^2}\]
    \[f'_y(x,y) = - 2xy \sin(xy^2) - \frac{2y}{x^2 + y^2}\]
    不难利用各一元函数的连续性,复合函数的连续性以及连续函数四则运算的连续性,可知这两个偏导数连续,故$f(x,y)$是可微的,利用可微函数的性质,我们有
    \[
    \begin{aligned}
        f'_{\ve{n}}(x,y) = \text{grad}f(x,y) \cdot \ve{n} &= \frac{\sqrt{3}}{2}\parameter{-y^2 \sin(xy^2) - \frac{2x}{x^2 + y^2}} + \frac{1}{2}\parameter{2xy \sin(xy^2) + \frac{2y}{x^2 + y^2}}\\
        &= \parameter{xy - \frac{\sqrt{3}}{2} y^2}\sin(xy^2) + \frac{y - \sqrt{3} x}{x^2 + y^2}
    \end{aligned}    
    \]

    (2)对于$x >0,y > 0$的情况
    
    同样我们先求偏导数
    \[g'_x(x,y) = y x^{y - 1} + (\ln y) y^x\]
    \[g'_y(x,y) = (\ln x) x^y + x y^{x - 1}\]
    利用和(1)相同的方法,我们有$g(x,y)$可微,故
    \[
        \begin{aligned}
            g'_{\ve{n}}(x,y) = \text{grad}g(x,y) \cdot \ve{n} &= \frac{\sqrt{3}}{2}( y x^{y - 1} + (\ln y) y^x) - \frac{1}{2}((\ln x) x^y + x y^{x - 1})\\
        \end{aligned}        
    \]
    对于$x = 0,y > 0$的情况,则$g(0,y) = 1$,
    \[
    \begin{aligned}    
        g'_{\ve{n}}(0,y) &= \lim_{t \to 0+}\frac{g((0,y) + \ve{n}t) - g(0,y) }{t}\\
        &= \lim_{t \to 0+} \frac{e^{(y - \frac{t}{2})\ln (\frac{\sqrt{3}}{2}t)} + e^{\frac{\sqrt{3}}{2}t \ln(y - \frac{t}{2})} - 1}{t}\\
    \end{aligned}    
    \]
    由于利用洛必达法则
    \[\lim_{t \to 0+} \frac{e^{\frac{\sqrt{3}}{2}t \ln(y - \frac{t}{2})} - 1}{t} = \frac{\sqrt{3}}{2}\ln y\]

    而对
    \[\lim_{t \to 0+} \frac{e^{(y - \frac{t}{2})\ln (\frac{\sqrt{3}}{2}t)}}{t}\]
    我们做变量代换$x = \ln\parameter{\frac{\sqrt{3}}{2}t}$,则化为
    \[\lim_{x \to -\infty} \frac{\sqrt{3}}{2} e^{(y - 1)x - \frac{\sqrt{3}}{3} e^x x}\]
    \begin{itemize}
        \item 当$y > 1$时,上述极限为$0$
        \item 当$y = 1$时,上述极限为$\frac{\sqrt{3}}{2}$
        \item 当$0<y < 1$时,上述极限不存在(正无穷)
    \end{itemize}
    故
    \[g'_{\ve{n}}(0,y) = \begin{cases}
        \frac{\sqrt{3}}{2}\ln y ,&y > 1\\
        \frac{\sqrt{3}}{2}\ln y + \frac{\sqrt{3}}{2},& y = 1\\
        \text{不存在},&0 < y < 1\\
    \end{cases}\]

    对于$x > 0,y = 0$,$\ve{n}$方向上的任何一个点都在函数定义域外,讨论方向导数无意义.

    同理对于$x,y$均为$0$的情况,不难发现这个方向上的任何一个点都在定义域外,从而讨论方向导数也是无意义的.
\end{exercise}

\begin{exercise}{3}
    \[f'_x(x,y,z) = g'_x(e^{x^2 y},z)e^{x^2 y} \cdot 2xy +  h'_x(\tan \frac{x}{z} + y) \frac{1}{\cos^2 \frac{x^2}{z^2}} \frac{1}{z}\]
    \[f'_y(x,y,z) = g'_y(e^{x^2 y},z) e^{x^2 y} \cdot x^2 + +  h'_y(\tan \frac{x}{z} + y)\]
    \[f'_z(x,y,z) = g'_z(e^{x^2 y},z) - h'_z(\tan \frac{x}{z} + y) \frac{1}{\cos^2 \frac{x^2}{z^2}} \frac{x}{z^2}\]
\end{exercise}

\begin{exercise}{4}
    (1)
    \[du = u'_x\, dx + u'_y\, dy = 2x \cos(x^2 + y^2)\, dx + 2y \cos(x^2 + y^2) \, dy\]

    (2)    
    \[dv = v'_x\, dx + v'_y\, dy = \frac{1}{\sqrt{1 - (\frac{x}{x^2 + 3y^2})^2}} \cdot \frac{-x^2 + 3y^2}{(x^2 + 3y^2)^2}\, dx - \frac{1}{\sqrt{1 - (\frac{x}{x^2 + 3y^2})^2}} \cdot \frac{6 xy}{(x^2 + 3y^2)^2}\, dy \]

    (3)
    \[dw = w'_x \,dx + w'_y \,dy + w'_z \,dz = (\ln 2)\cdot 2^x y^3 \cosh(2z + 1)\, dx + 3\cdot  2^x y^2 \cosh(2z + 1)\, dy +2\cdot  2^x y^3 \sinh(2z + 1)\, dz\]
\end{exercise}

\begin{exercise}{5}
    \[z'_x = \cos x \sin y - e^{x + y},z'_y = \cos y \sin x - e^{x + y}\]
    从而
    \[z'_x(1,1) = \cos 1 \sin 1 - e^{2},z'_y(1,1) = \cos 1 \sin 1 - e^{2}\]    
    故我们得到了两个切向量
    $(1,0,\cos 1 \sin 1 - e^{2})$和$(0,1,\cos 1 \sin 1 - e^{2})$
    利用$\RR^3$中向量的叉乘,我们有
    \[\ve{n} = \det \begin{bmatrix}
        i&j&k\\
        1&0&\cos 1 \sin 1 - e^{2}\\
        0&1&\cos 1 \sin 1 - e^{2}\\
    \end{bmatrix} = (e^2 -\cos 1 \sin 1,e^2 -\cos 1 \sin 1,1)\]
    根据这个法向量,我们立刻可以得到切平面
    \[P: (e^2 -\cos 1 \sin 1)x + (e^2 -\cos 1 \sin 1)y + z = 0\]
\end{exercise}

\begin{exercise}{6}
    对于$(0,0)$邻域$U((0,0),\delta)$,$\forall \delta > 0$中的的点$(x,y)$,不妨限定$x,y < \frac{1}{2}$
    \[f'_x(x,y) = 2x \sin \frac{1}{x^2 + y^2} - \frac{2x}{x^2 + y^2} \cos \frac{1}{x^2 + y^2} \]
    利用绝对值三角不等式
    \[|f'_x(x,y)| \geq \abs{\frac{2x}{x^2 + y^2} \cos \frac{1}{x^2 + y^2}} - 1 \]
    我们取这个邻域中的点$(\sqrt{\frac{1}{2k\pi}},0)$,其中$k$取$\bracket{\frac{1}{2\delta^2 \pi}} + m$,$m = 1,2,\ldots$,那么我们就有
    \[|f'_x(x,y)| \geq 2 \sqrt{2k\pi} - 1\]
    由于这里的$k$可以任意大,故这个偏导数是无界的\\

    利用对称性,我们同样可以证明偏导数$f'_y(x,y)$也是无界的\\

    然而,对于$r = (x,y)$在$(0,0)$的邻域内,我们通过变量代换$\|r\| = \sqrt{x^2 + y^2}$,也就有
    % 以及$(x,y)\to (0,0)$时有$\|r\| \to 0$可以推出
    % \[\lim_{(x,y) \to (0,0)} f(x,y) = \lim_{\|r\| \to 0} \|r\|^2 \sin \frac{1}{\|r\|^2} = 0 = f(0,0)\]
    % 而
    \[f(x,y) = \|r\|^2 \sin \frac{1}{\|r\|^2} = o(\|r\|)\]
    也就是说
    \[f(h,k) - f(0,0) = 0 h + 0 k + o(\sqrt{h^2 + k^2})\]
    即$f$在$(0,0)$处可微
\end{exercise}

\begin{exercise}{7}
    \[f'_x(0,0,0) = \lim_{t \to 0} \frac{f(t,0,0) - f(0,0,0)}{t} = \lim_{t \to 0} \frac{0}{t} = 0 \]
    其中最后一个等式利用了洛必达法则

    同理我们可利用对称性得知$f'_y(0,0,0),f'_z(0,0,0)$都存在.\\

    然而,当我们取方向向量$e = (\frac{\sqrt{3}}{3},\frac{\sqrt{3}}{3},\frac{\sqrt{3}}{3})$时,我们的
    \[\frac{\partial f}{\partial e}(0,0,0) = \lim_{t \to 0} \frac{f(\frac{\sqrt{3}}{3}t,\frac{\sqrt{3}}{3}t,\frac{\sqrt{3}}{3}t) - f(0,0,0)}{t} = \lim_{t \to 0} \abs{\frac{1}{t}} = +\infty\]
    倒数第二个等号是因为这个式子的分子在$t > 0$时为$1$,在$t < 0$时为$-1$,故最终极限不存在.

    由于$(0,0)$处存在一个方向导数不存在,故$f$在$(0,0)$一定不可微.

\end{exercise}



\end{CJK}
\end{document}

\begin{itemize} 
\end{itemize}