\documentclass[11pt,hyperref,a4paper,UTF8]{ctexart}
\usepackage[left=2.50cm, right=2.50cm, top=2.50cm, bottom=2.50cm]{geometry}
\usepackage[unicode=true,colorlinks,urlcolor=blue,linkcolor=blue,bookmarksnumbered=true]{hyperref}
\usepackage{graphicx} % Required for inserting images
\usepackage{ctex}

\usepackage{amsfonts}
\usepackage{amssymb}
\usepackage{amsmath}
\usepackage{amsthm}
% 导入 xparse 宏包以支持 LaTeX3 语法
\usepackage{xparse}
\usepackage{pgfplots}
% 用于插入带有坐标轴、标签和曲线的图像

% \begin{tikzpicture}
%     \begin{axis}[
%       xlabel=$x$,
%       ylabel=$f(x)$,
%       axis lines=middle,
%       xmin=-5, xmax=5,
%       ymin=-2, ymax=8,
%       width=\textwidth,
%       height=8cm
%     ]
%     \addplot[blue,domain=-3:3] {x^2};
%     \end{axis}
%   \end{tikzpicture}
\usepackage{tikz}
% 用于绘制一般的图像

% \begin{tikzpicture}[scale=0.8]
%     \draw[->] (-4,0) -- (4,0) node[right] {$x$};
%     \draw[->] (0,-1) -- (0,9) node[above] {$f(x)$};
%     \draw[domain=-3:3,smooth,variable=\x,blue] plot ({\x},{\x^2});
%   \end{tikzpicture}

\newtheorem{theorem}{Theorem}[subsection]
\newtheorem{lemma}{Lemma}[subsection]
\newtheorem{corollary}{corollary}[subsection]
\newtheorem{example}{Example}[subsection]
\newtheorem{definition}{Definition}[subsection]
% 为了证明中可以使用中文,后续定义证明时使用cproof而不是proof
\newenvironment{cproof}{%
{
    \textbf{Proof\\}
    }
}{
%   \hfill $\square$ 添加结束符号
%   \par\bigskip 可选的垂直间距
}
\newenvironment{exercise}[1]{%
{\textbf{Exercise #1} \\ 
    }
}{
  \hfill $\square$ 
  \par\bigskip 
}

\newenvironment{solution}{%
{
    \textbf{Solution\\}
    }
}{
  \hfill $\square$ 
  \par\bigskip 
}


% \newenvironment{identification}{%
% \heiti{定义}\kaishu
% }{%
% %   \hfill $\square$ 添加结束符号
% %   \par\bigskip 可选的垂直间距
% }

\newcommand{\RR}{\mathbb{R}}
\newcommand{\NN}{\mathbb{N}}
\newcommand{\CC}{\mathbb{C}}
\newcommand{\QQ}{\mathbb{Q}}
\newcommand{\ZZ}{\mathbb{Z}}
\newcommand{\FF}{\mathbb{F}}
\newcommand{\PP}{\mathbb{P}}
% 简化各种常见数的集合

\newcommand{\parameter}[1]{\left(#1\right)}

\newcommand{\bracket}[1]{\left[#1\right]}

\newcommand{\abs}[1]{\left|#1\right|}
% 各种自动变化大小的括号的简化

\newcommand{\ve}{\boldsymbol}
% 为了适应David C Lay线性代数中,简化斜体+粗体向量的书写

\newcommand{\base}{\mathcal}

\newcommand{\tb}{\textbf}

\newcommand{\col}{\text{Col}}

\newcommand{\row}{\text{Row}}
\newcommand{\nul}{\text{Nul}}
\newcommand{\spans}{\text{Span}}
\newcommand{\proj}{\text{proj}}
\newcommand{\adj}{\text{adj.}}
\newcommand{\rank}{\text{rank}}
\newcommand{\range}{\text{range}}
\newcommand{\n}{\text{null}}
\newcommand{\tr}{\text{tr}}
\newcommand{\sign}{\text{sign}}
\newcommand{\perm}{\text{perm}}
% 简化粗体字体的书写

\newcommand{\f}[2]{\frac}
\newcommand{\df}[2]{\dfrac}

\newcommand{\ip}[1]{\left<#1\right>}

\newcommand{\pa}{\paragraph}
\newcommand{\spa}{\subparagraph}
\newcommand{\se}{\section}
\newcommand{\sse}{\subsection}
\newcommand{\ssse}{\subsubsection}

% \NewDocumentCommand{\vs}{m m m}{
%     \ve{#1}_{#2},\cdots,\ve{#1}_{#3}
% }
% % 快速书写一个向量组,第一个参数为向量名称,后两个为首末角标

% % $\vs{b}{1}{n} $这是多参数命令的使用示例

% \NewDocumentCommand{\cvs}{m m m m m}{
%     #1_{#4}\ve{#2}_{#4} #3 \cdots #3 #1_{#5}\ve{#2}_{#5}
% }
% % 快速书写一个向量线性组合,第一个参数为系数,第二个参数为向量名称,第三个参数为运算符,后两个参数为角标

\NewDocumentCommand{\size}{m m}{
    #1\times#2
}


\title{习题十一}
\author{徐海翁}
\date{2024.5.15}

\begin{document}

\begin{CJK}{UTF8}{gkai}

\maketitle
% \tableofcontents

\begin{exercise}{3}
(1)如果令积分顺序为$x,y$,那么积分域分别为
\[\left[-\sqrt{4 - y^2},\varphi(y)\right],\left[\frac{-2 - \sqrt{19}}{5},1\right]\]

其中
\[\varphi(y) = \begin{cases}
    \sqrt{4 - y^2},& \frac{-2 + \sqrt{19}}{5} < y \leq 1\\
    2y + 1, &\frac{-2 - \sqrt{19}}{5} \leq y \leq \frac{-2 + \sqrt{19}}{5}
\end{cases}\]

(2)如果令积分顺序为$z,y,x$,那么积分域分别为
\[\left[-x - y, 1 - x - y\right],\left[-x ,1 - x\right],\left[0,1\right]\]

(3)如果令积分顺序为$y,x$,那么积分域分别为
\[\left[0,\sin x\right], \left[0,\frac{2\pi}{3}\right]\]
\end{exercise}

\begin{exercise}{4}
    (1)
    \[= \int_{0}^{1}\int_{W_y} f(x,y)\, dx dy\]

    其中
    \[W_y = \begin{cases}
        [y^2,\frac{1 - \sqrt{1 - 4y^2}}{2}]\cup [\frac{1 + \sqrt{1 - 4y^2}}{2},1],& 0 \leq y \leq \frac{1}{2}\\
        [y^2,1],& \frac{1}{2} < y \leq 1\\
    \end{cases}\]

    (2)
    \[= \int_{0}^{2} \int_{W_y} f(x,y)\, dx dy\]

    其中
    \[W_y = \begin{cases}
        [1 , 3 - y],& 1 \leq y \leq 2\\
        [3, 3 - y],& 0 \leq y < 1\\
    \end{cases}\]
\end{exercise}

\begin{exercise}{5}
    (1)
    \[\iint_D (x^2 + y)\, d(x,y) = \int_{-2}^{6} dy \int_{\frac{y^2}{4}}^{y + 3}(x^2 + y)\, dx = \int_{-2}^{6}\parameter{- \frac{1}{192}y^6 + \frac{1}{12}y^3 + 4 y^2 + 12y + 9}\,dy = \frac{8000}{21}\]

    (2)

    \[\iint_D \sin(x + 2y)\, d(x,y) = \int_{1}^{\pi}\, dx \int_{x - \pi}^{x + \pi} \parameter{-\frac{1}{2}} d(\cos(x+ 2y)) = \int_{1}^{\pi}\, dx \cdot 0 = 0\]
\end{exercise}

\begin{exercise}{6}
    \[E_1 = \{(x,y)| x\text{为}[0,1]\text{上的有理数},y \in [0,1]\}\]
    \[E_2 = \{(x,y)| x\text{为}[0,1]\text{上的无理数},y \in [0,1]\}\]

    利用边界点的定义,以及有理数和无理数都具有的稠密性,我们知道$\forall p = (x,y) \in [0,1]\times [0,1]$,$\forall \delta > 0$,
    \[U(p,\delta) \cap  E_1 \neq \varnothing\]
    \[U(p,\delta) \cap  (\RR^2 \setminus E_1) \neq \varnothing\]
    \[U(p,\delta) \cap  E_2 \neq \varnothing\]
    \[U(p,\delta) \cap  (\RR^2 \setminus E_2) \neq \varnothing\]

    因此$E_1,E_2$的边界点集都是$[0,1]\times[0,1]$,显然都不是零测集,因此$E_1,E_2$不是Jordan可测集.

    但是$E_1 \cap E_2 = \varnothing$,其边界点集也是空集,故$E_1 \cap E_2$可测,而$E_1 \cup E_2 = [0,1]\times[0,1]$,也显然可测.
\end{exercise}

\begin{exercise}{7}
    (1)是零集,我们可以按照Cantor三角列出$[0,1]$中的有理数(可数无穷个),对于每个有理数$r$,我们使用闭方块
    \[[r - \frac{1}{2} \frac{\varepsilon}{2^n}, r + \frac{1}{2} \frac{\varepsilon}{2^n}] \times [0,1]\]

    进行覆盖,其中$n$为这个有理数的序号,这样所有的闭方块覆盖了这个集合中所有的点,而它们的总面积为
    \[\sum_{i = 1}^{+\infty} Vol(Q_i) = \lim_{n \to +\infty} \sum_{i = 1}^{n} \frac{\varepsilon}{2^i} \leq \varepsilon\]

    (2)不是零集

    (3)是零集
\end{exercise}

\begin{exercise}{8}
    (1)正确

    (2)正确

    (3)正确

    (4)正确

    (5)不正确[考虑之前写过的$m + \sqrt{2} n$等形式]

    (6)正确

    (7)正确

    (8)不正确[参考(5)]

    (9)不正确[显然可以相等]

    (10)不正确[定义]
\end{exercise}

\end{CJK}
\end{document}

\begin{itemize} 
\end{itemize}